\section{Практика} \marginpar{\framebox{02.04.2014}}
\begin{tsk}
\begin{tikzpicture}
\node[] (s1) {4.Саша};
\node[right=of s1] (s2) {3.Аркаша};
\node[above=of s1] (s3) {2.Маньяк 1};
\node[above=of s2] (s4) {1.Маньяк 2};
\path[->]
(s1) edge node[above]{2/3} (s2)
(s2) edge[bend left] node[below]{1/2} (s1)
(s1) edge node[left]{1/6} (s3)
(s1) edge[loop left] node[left]{1/6} (s1)
(s2) edge[loop right] node[right]{1/3} (s2)
(s2) edge node[right]{1/6} (s4);
\end{tikzpicture}\\
Питання : 
\begin{equation}
h_i = \mP\set{\xi_\infty = 1\setminus \xi_0 = i}
\end{equation}
\begin{eqnarray}
h_1 &=& 1\\
h_2 &=& 0\\
h_3 &=& \cfrac16 \cdot h_1 + \cfrac16 \cdot h_3 + \cfrac23 \cdot h_4\\
h_4 &=& \cfrac12 \cdot h_3 + \cfrac16 h_2 + \cfrac13 \cdot h_4
\end{eqnarray}
Отримали: $h_3 = \cfrac12;h_4 = \cfrac38$\\
Тепер знайдемо середній час життя:
\begin{eqnarray}
m_1 &=& 0\\
m_2 &=& 0\\
m_3 &=& 1 + \cfrac16 m_1 + \cfrac16 m_3 + \cfrac23 m_4 \\
m_4 &=& 1 + \cfrac12 m_3 + \cfrac12 m_2 + \cfrac13 m_4
\end{eqnarray}
Отримали: $m_3 = m_4 = 6$
\end{tsk}
\begin{tsk}
\begin{tikzpicture}[node distance= 3cm]
\node[circle,draw,label=below:Дом] (s1) {0};
\node[circle,draw,label=below:Нічний клуб,right=of s1] (s2) {};
\node[circle,draw,label=below:Міліція,right=of s2] (s3) {n};
\path[-]
(s1) edge (s2) edge (s3)
;
\end{tikzpicture}\\
\begin{eqnarray}
&h_0 = 1\\
&h_n = 0\\
k=1,\ldots,n & h_k = \cfrac12 h_{k+1}+ \cfrac12 h_{k-1}
\end{eqnarray}
\begin{equation}
h_k = 1 - \cfrac kn
\end{equation}
\end{tsk}
\begin{tsk}
Та сама задача, але скільки часу займе її прогулянка?\\
Тобто, знайдемо $m_k$
\begin{eqnarray}
m_0 &=& 0\\
m_n &=& 0\\
m_k &=& 1 + \cfrac12 m_{k+1}  + \cfrac12 m_{k-1}
\end{eqnarray}
Розв’язавши це добро можна отримати:
\begin{equation}
m_k = k - k^2 + k \cb{n-1}
\end{equation}
\end{tsk}
\subsection{Домашнє завдання}
\begin{tsk}
Розв’язати обидві ці задачі, якщо ймовірність піти направо дорівнює $p$, а наліво $q$. Звісно, $p+q=1$.
\end{tsk}
\begin{tsk}
%переписати у Ганни
\end{tsk}
