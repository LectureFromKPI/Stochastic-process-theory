\documentclass[a5paper,12pt,oneside,ukrainian]{book}
\usepackage[T2A]{fontenc}
\usepackage[utf8]{inputenc}
\usepackage{amssymb,amsmath,amsthm}
\usepackage{vmargin}
\usepackage{setspace}
\usepackage[ukrainian]{babel}
\usepackage{tikz}
\usepackage[unicode,colorlinks]{hyperref}
\usepackage{demath}
\usetikzlibrary{patterns,positioning}
\setlength{\parindent}{1.25cm}
\begin{document}
\tableofcontents
\numberwithin{equation}{section}
\chapter{Теорія випадкових процесів}
\section{Елементи аксіоматики А. Н. Колмогорова}\marginpar{\framebox{12.02.2014}}
Була запропонована у 1933 році у праці "Основні поняття теорія імовірностей" німецькою мовою. \\
Основним поняттям аксіоматики є $\cb{\Omega,\iF,\mP}$, де $\Omega$ - простір ймовірнісних подій, $F$ - $\sigma$-алгебра подій. $\mP$ - ймовірнісна міра.
\begin{exs}[Двократне підкидання монетки]\label{tr:1:1}
\begin{equation*}
\Omega=\set{RR=\nw1,RG=\nw2,GR=\nw3,GG=\nw4}
\end{equation*}
Множина випадкових подій : 
\begin{equation*}
\set{\emptyset,\set{RR},\set{RG},\set{GR},\set{GG},\set{RR,RG},\set{RR,GG},\ldots,\Omega}
\end{equation*}
$\emptyset$ - неможлива подія. \\
$\Omega$ - достовірна подія. \\
Отже, в нас виникло 16 випадкових подій. \\
\begin{equation*}
\iF=2^\Omega
\end{equation*}
\end{exs}
Що ми хочемо відносно \iF (класу всіх випадкових подій).
\begin{itemize}
\item $\Omega\in\iF$
\item $A\in\iF\rightarrow A\in\iF$
\item $\set{A_1,\ldots}\in\iF\rightarrow \bcupl_{i=1}^\infty \in\iF$
\end{itemize}
Чому не можна завжди брати в якості $\iF=2^\Omega$?
\begin{exs}[Контрприклад від G. Vitali]
Кидаємо матеріальну точку.
\begin{equation}
x+y = (x+y)\md 1
\end{equation}
Розглянемо на відрізку відношення: $x\sim y \leftrightarrow y-x\in\mQ$
\begin{eqnarray*}
A_0 &=& \set{x\in\bb{0,1}:x~0}\\
A_{\frac{\sqrt3}2} &=& \set{x\in\bb{0,1}:x~\frac{\sqrt3}2}
\end{eqnarray*}
Розглянемо деяку множину $E$ - множина, яка включає рівно по одному представнику з кожного класу еквівалентності. \\
Для будь-якого раціонального числа $\alpha\in\mQ\cap\bcb{0,1}$ для нього існує $E_\alpha = E+\alpha$.\\
Подія $E$ - точка потрапила у множину $E$.\\
Можна отримати деякі властивості:
\begin{itemize}
\item $\mP\set{E} = \mP\set{E_\alpha},\alpha\in\mQ$\\
\item $E_\alpha\cap E_\beta = \emptyset,\alpha\neq\beta,\alpha,\beta\in\mQ$\\
\item $\bcupl_{\alpha\in\mQ} E_\alpha=\bcb{0,1}$
\end{itemize}
Отже, ми отримали, що:
\begin{eqnarray}
&\mP\set{E} = \mP\set{E_\alpha}=p\\
&1=\mP\set{\bcb{0,1}}=\mP\set{\bcupl_{\alpha\in\mQ}E_\alpha}=\suml_{\alpha\in\mQ}\mP\set{E_\alpha}=\suml_{n=1}^\infty p
\end{eqnarray}
Отримали протиріччя. Помітимо те, що тут використовувалася незліченність відрізку $\bcb{0,1}$
\end{exs}
$\mP:\iF\to\bcb{0,+\infty}$ - має бути мірою на \iF\\
І є лише дві вимоги, яким вона має задовольняти:
\begin{itemize}
\item $\sigma$-адитивність: $\forall A_1,\ldots; A_i\cap A_j=\emptyset,i\neq j \rightarrow \mP\set{\bcupl_{i=1}^\infty A_i} = \suml_{i=1}^\infty \mP\set{A_i}$
\item $\mP\set{\Omega} = 1$
\end{itemize}
\begin{nasl}
$\mP\set{\emptyset}=0$
\end{nasl}
\begin{proof}
$1=\mP\set{\Omega}=\mP\set{\Omega\cup\emptyset} = \mP\set{\Omega}+\mP\set{\emptyset} \rightarrow \mP\set{\emptyset}=0$
\end{proof}
\begin{nasl}
$A\subset B \rightarrow \mP\set{B} \geq \mP\set{A}$
\end{nasl}
\begin{proof}
$\mP\set{B} = \mP\set{A\cup\cb{B\setminus A}} = \mP\set{A} + \mP\set{B\setminus A } \geq \mP\set{A}$\\
\end{proof}
\begin{nasl}
$\forall A:\mP\set{A} \leq 1$
\end{nasl}
\begin{proof}
$A \subset \Omega \rightarrow \mP\set{A} \leq \mP\set{\Omega} = 1$
\end{proof}
Випадкова величина є функцією $\xi:\Omega\to\mR$. Більш широко це позначає, що вона вимірно діє на двох вимірних просторах
\begin{equation}
\cb{\Omega,\iF}\xrightarrow{\text{вим.}}\cb{\mR,\iB\cb{\mR}}:\forall B\in\iB\cb{\mR}; \xi^{-1}\cb{B}\in\iF
\end{equation}
Тобто, дійсно існує ймовірність:
\begin{equation}
\mP\set{\xi\in B} = \mP\set{\omega\in\Omega:\xi(\omega)\in B} = \mP\set{\xi^{-1}\cb{B}}
\end{equation}
Тоді, можна стверджувати, що:
\begin{equation}
\exists  \mP\set{\xi\in\cb{-\infty,x}}=\mP\set{\xi<x} = F_\xi(x)
\end{equation}
Випадкові величини можна поділити на типи:
\begin{itemize}
\item {\bf Дискретні випадкові величини} - множина значень скінчена або зліченна.
\item {\bf Неперервні випадкові величини} - функція розподілу неперервна.
\item {\bf Змішані випадкові величини} - випадкові величини змішаного типу.
\end{itemize}
\begin{teor}\label{tr:1:2}
$\xi$ є неперервною $\leftrightarrow$ $\forall x\in\mR:\mP\set{\xi=x}=0$
\end{teor}
\begin{proof}
$\mP\set{\xi=x}=\mP\set{\bcapl_{n=1}^\infty \xi\in\bcb{x,x+\frac1n}} = \liml_{n\to\infty} \mP\set{\xi\in\bcb{x,x+\frac1n}} = \liml_{n\to\infty} F_\xi(x+\frac1n) -F_\xi(x) = F_\xi(x+0)-F_\xi(x)$
\end{proof}
\begin{teor}
$\xi$ є неперервною $\leftrightarrow$ $\forall B\subset\mR$ - множина, яка є зліченною або скінченною, виконується $\mP\set{\xi\in B} =0$.
\end{teor}
\begin{proof}
Доведемо еквівалентність з попереднім критерієм \ref{tr:1:2}.\\
Нехай $\forall B\subset\mR:\mP\set{\xi\in B} =0$\\
Тоді $\forall x_j\in\mR;\mP\set{\xi=x_j} = \mP\set{\xi\in\set{x_j}} = 0 $\\
З множини в точку - очевидно.\\
Нехай $\forall x_j \in \mR: \mP\set{\xi=x_j}=0$
$B=\set{x_1,x_2,\ldots}$\\
$\mP\set{\xi\in B} = \suml_{j=1}^\infty \mP\set{\xi=x_j} = \suml_{j=1}^\infty 0 = 0$
\end{proof}
Випадкова величина називається {\bf абсолютно неперервною величиною}, якщо
\begin{equation}
\forall B\subset\mR:\lambda\cb{B} =0\rightarrow\mP\set{\xi\in B}=0
\end{equation}
Чому в такому випадку існує щільність розподілу?\\
$\cb{\Omega,\iF,\mP} \xrightarrow{\xi} \cb{\mR,\iB\cb{\mR},\iPx}$\\
$\iPx$ - образ міри $\mP$ на \cb{\mR,\iB\cb{\mR}} породжена випадковою величиною $\xi$\\
$\forall B\in\iB\cb{\mR}:\iPx\cb{B} = \mP\set{\xi^{-1}\cb{B}} = \mP\set{\xi\in B}$
\begin{exs}[Будування образу міри]
Умова задачі з попереднього прикладу \ref{tr:1:1}.\\
Зрозуміло, що $\mP\set{\nw1} = \mP\set{\nw2} = \mP\set{\nw3} = \mP\set{\nw4} = \frac14$\\
%вставити малюнок 3B
\begin{tikzpicture}
\node[circle,draw,label=left:$\frac14$](c1){РР} ;
\node[circle,draw,below=0.3cm of c1,label=left:$\frac14$](c2){РГ};
\node[circle,draw,below=0.3cm of c2,label=left:$\frac14$](c3){ГP};
\node[circle,draw,below=0.3cm of c3,label=left:$\frac14$](c4){ГГ};
\node[label=right:$\frac14$,right=of c1](g0){0};
\node[label=right:$\frac12$,below=1.4cm of g0] (g1){1};
\node[label=right:$\frac14$,right=of c4] (g2){2};
\path[->]
(c1) edge (g0)
(c2) edge (g1)
(c3) edge (g1)
(c4) edge (g2);
\end{tikzpicture}\\
$\xi$ - кількість гербів. \\
$\iPx(0)=\frac14,\iPx(1)=\frac12,\iPx(2)=\frac12$
\end{exs}
$\iPx$ часто називається {\bf розподілом випадкової величини $\xi$}\\
Якщо $\xi$ - абсолютно неперервна величина, то виконується:
\begin{equation}
\lambda(B)=0\Rightarrow \mP\set{\xi\in B} = 0 \Rightarrow \iPx(B) = 0
\end{equation}
Міра $\iPx$ для абсолютно неперервних випадкових величин є абсолютно неперервною відносно міри Лебега.\\
Отже, можна використовувати теорему Радона-Нікодима:
\begin{equation}
\exists f\in \mL_1\cb{\mR,\lambda}:\iPx(B)=\intl_B f(t)\dt
\end{equation}
Оскільки це виконувалося для будь-який $B$ ми можемо підставити туди будь-що. Підставимо $\cb{-\infty,x}$\\
\begin{equation}
F_\xi(x) = \mP\set{\xi<x} = \intl_{-\infty}^x f(t)\dt
\end{equation}
Отже, $f(t)$ - щільність розподілу.\\
Неперервні випадкові величини, які не є абсолютно неперервними називаються {\bf сингулярними}.\\
\section{Випадкові вектори}\marginpar{\framebox{26.02.2014}}
\textbf{Випадковий вектор} - це набір з $n$ випадкових величин, які задані на спільному ймовірнісному просторі $\Omg$.  \\
$\vxi = \vect{\xi_1\\\xi_1\\\vdots\\\xi_n}$\\
$\xi_i:\Omg\to\mR$,$\xi_i$ - вимірна відносно обох $\sigma$-алгебр $\iF$, $\iB\cb{\mR}$\\
Але це означення досить погане, тому введемо інше:\\
$\vxi$ - це випадкова величина, яка $\vxi:\Omg\to\mR^n$, яка вимірна відносно $\sigma$-алгебр $\iF$ та $\iB\cb{\mR^n}$. \\
\begin{teor}
Два попередні означення еквівалентні.
\end{teor}
\begin{proof}
Нехай виконується:
\begin{equation*}
\vxi:\cb{\Omg,\iF}\xrightarrow{\text{вим.}}\cb{\mR^n,\iB\cb{\mR^n}}
\end{equation*}
Введемо функцію: $\pi:\mR^n\to\mR$ така, що $\pi_i(\vx) = x_i$\\
$\xi_i(\omg) = \pi_i\cb{\vxi(\omg)} = \pi \circ \vxi$ - вимірна, як композиція вимірних.\\
Залишилося довести, що є перехід з першого в друге.\\
Про вектор відомо лише те, що кожна його координата є вимірною функцією. \\
Потрібно довести, що $\vxi(\omg)$ вимірна відносно $\iF$ та $\iB\cb{\mR^n}$.\\
Тобто, потрібно довести, що 
\begin{equation}
\forall B\in \iB\cb{\mR^n}:\vxi^{-1} \cb{B} \in\iF
\end{equation}
Скористаємося методом гарних множин. Візьмемо такі множин з $\iB\cb{\mR^n}$, для яких $\vxi^{-1} \cb{B} \in\iF$. Назвемо цю множину множин $C$. Очевидно, що $C\subset \iB$. Також нам відомо, що $B_1\times B_2\times\ldots\times B_n\in C,B_i\in\iB\cb{\mR},\forall i$\\
Доведемо це.\\
$\underbrace{\vxi^{-1}\cb{B_1\times B_2\times\ldots\times B_n}}_{\text{брус}} = \set{\omg\in\Omg:\vxi(\omg) \in B_1\times B_2\times\ldots\times B_n} =$ \\ $= \set{\omg\in\Omg:\forall i\in\set{1,\ldots,n} \xi_i \in B_i}  =\bcapl_{i=1}^n \set{\omg\in\Omg:\xi_i(\omg)\in B_i} \in \iF$ \\
Чи можна стверджувати, що $\mR^n\in C$?
\begin{equation}
\vxi^{-1}\cb{\mR^n} = \Omg\in\iF
\end{equation}
Отже, таки так. \\
\begin{equation}
B_1,B_2,\ldots\in C\Rightarrow \bcupl_{i=1}^\infty B_i \in C
\end{equation}
Доведемо це:
\begin{equation}
\vxi^{-1}\cb{\bcupl_{i=1}^\infty B_i} = \bcupl_{i=1}^\infty \vxi^{-1}\cb{B_i} \in C
\end{equation}
Нам залишилася необхідною лише одна властивість:
\begin{equation}
B\in C\Rightarrow \overline B \in C
\end{equation}
І це також доведемо, хоча й очевидно, наче:
\begin{equation}
\vxi^{-1}\cb{\overline B} = \overline{\vxi^{-1}\cb{B}}\in\iF
\end{equation}
З отриманих властивостей можна визначити, що наша $C$ є $\sigma$-алгеброю та містить всі бруси. Отже, вона містить і $\sigma$-алгебру породжену брусами, а це є $\iB\cb{\mR^n}$. Отже, $\iB\set{\mR^n}\subset C$. Отримали, що $C=\iB\cb{\mR^n}$.
\end{proof}
\section{Характеристики випадкових векторів}
$\vxi = \vect{\xi_1\\\xi_2\\\vdots\\\xi_n}$
\subsection{Математичне сподівання}
$\mEt{\vxi} = \vect{\mEt{\xi_1}\\\mEt{\xi_2}\\\vdots\\\mEt{\xi_n}}$
\subsection{Кореляційна матриця}
Також має назву \textbf{коваріаційної} або \textbf{дисперсійної матриці}.
\begin{equation}
\iDx = \mEt{\cb{\vxi-\mEvx}\cb{\vxi-\mEvx}^\ast}
\end{equation}
Властивості цієї матриці:
\begin{enumerate}
\item $\iDx^\ast = \iDx$
\begin{proof}
$\iDx^\ast = \cb{\mEt{\cb{\vxi-\mEvx}\cb{\vxi-\mEvx}^\ast}}^\ast = \mEt{\cb{\vxi-\mEvx}^{\ast,\ast}\cb{\vxi-\mEvx}^\ast}=\iDx$
\end{proof}
\item \iDx - невід’ємно визначена.
\begin{proof}
$\cb{\iDx\vc,\vc} = \cb{\mEt{\vxi_0\vxi_0^\ast}\vc,\vc} = \mEt{\vxi_0^\ast\vc,\vxi_0\vc}\geq 0$
\end{proof}
\begin{war}
Подумати про те, коли буде досягнута рівність.
\end{war}
\item Будь-яка матриця, яка задовольняє дві попередні властивості обов’язково є кореляційною матрицею деякого вектору. 
\end{enumerate}
Також можна ввести формулу \textbf{взаємнодисперсійної матриці}:
\begin{equation}
C_{\vxi,\veta} = \mEt{\cb{\vxi - \mEx}\cb{\veta-\mEe}}
\end{equation}
\begin{enumerate}
\item $C_{\vxi,\veta} = C_{\vxi,\veta}^\ast$
\end{enumerate}
\begin{exs}
Розглянемо деяку матрицю: \\
$t_1,\ldots,t_n\geq0$\\
Розглянемо матрицю їх локальних мінімумів:
\begin{equation*}
||\min\cb{t_i,t_j}||_{i,j} = \begin{pmatrix}
t_1 & \min\cb{t_1,t_2} & \ldots \\
\min\cb{t_2,t_1} & t_2 & \ldots \\
\vdots &\vdots & \vdots
\end{pmatrix}
\end{equation*}
$H$ - гільбертовий простір. $x_1,\ldots,x_n\in H$
\begin{equation*}
G = \begin{pmatrix}
(x_1,x_1) & (x_1,x_2) & \ldots & (x_1,x_n) \\
(x_2,x_1) & \ldots\\
\vdots & \vdots & \vdots & \vdots \\
(x_n,x_1) & \ldots
\end{pmatrix}
\end{equation*}
Це \textbf{матриця Грама}, яка є ермітова та невід’ємно визначена.\\
Спробуємо побудувати деякий гільбертовий простір такий, щоб в ньому було легко знаходити скалярні добутки:\\
%замість 
Візьмемо простір $\mL_2\cb{0,+\infty}:x_i(t) = I_{\bb{0,t_i}}(t)$
\begin{equation*}
\cb{x_i,x_j} = \intzi I_{\bb{0,t_j}} I_{\bb{0,t_i}} \dt = \min\cb{t_i,t_j}
\end{equation*}
Отже, матриця справді кореляційна.
\end{exs}
\subsection{Перетворення \iDx та $C_{\vxi,\veta}$ при афінних перетвореннях}
Задані$\vxi,\iDx$ та афінне перетворення:
\begin{equation}
\veta = A\vxi + \vb
\end{equation}
Спробуємо знайти, як зміниться дисперсійна матриця при такому афінному перетворені:
\begin{multline}
\iDe = \mEt{\cb{\veta-\mEve}\cb{\veta-\mEve}^\ast} ={}\\{}= \mEt{\cb{A\vxi + \vb+\mEt{A\vxi + \vb}}\cb{A\vxi + \vb+\mEt{A\vxi + \vb}}^\ast} ={}\\{}= \mEt{\cb{A\vxi+\mEvx}\cb{A\vxi+\mEvx}^\ast} = A\iDx A^\ast
\end{multline}
\subsection{Гільбертовий простір $\mL_2\cb{\Omg,\mR^n}$}
Введемо такий гільбертовий простір:
\begin{equation}
\mL_2\cb{\Omg,\mC^n} = \set{\vxi:\cb{\Omg,\iF}\to\cb{\mC^n,\iB\cb{\mC^n}}:\mEt{|\xi_i|^2}<\infty,\forall\ifon}
\end{equation}
З нерівності Коші-Буняковського отримуємо, що:
\begin{equation}
\mEt{\xi_i\bar{\xi}_j}\leq \sqrt{\mEt{|\xi_i|^2}}\sqrt{\mEt{|\xi_j|^2}}< \infty
\end{equation}
Знайдемо в такому просторі скалярний добуток:
\begin{equation}\label{tr:3:2}
\cb{\vxi,\veta} = \mEt{\cb{\vxi,\veta}_{\mR^n}} = \mEt{\veta^\ast \vxi} = \mEt{\tr {\cb{\vxi,\veta^\ast}}}
\end{equation}
\section{Оптимальне лінійне оцінювання випадкових векторів}
$\vxi,\veta$ - випадкові вектори.\\
$\vxi\in\mL_2\cb{\Omg,\mC^m},\veta\in\mL_2\cb{\Omg,\mC^n}$\\
Вектор $\vxi$ спостерігається, а потрібно знайти оцінку $\hveta$. \\
Будемо шукати оцінку в такому вигляді: $\hveta = A\vxi +\vb$.\\
Ми розглядаємо критерій мінімуму середньоквадратичного відхилення:
\begin{equation}
\min\set{\mEt{\nr{\veta-\hveta}^2}}\\
\end{equation} 
Шукаємо $A\in\mathcal{MAT}_{n\times m},\vb\in \mC^n$, щоб 
\begin{equation}
\mEt{\nr{\veta - \cb{A\vxi+\vb} }^2} = \min\set{\mEt{\nr{\veta - \cb{C\vxi-\vd}}^2}}
\end{equation}
\begin{teor}[Лема про перпендикуляр у гільбертовому просторі]\label{tr:3:1}
$H$ - гільбертовий простір. $L\subset H$ - підпростір. $y\not\in L$.\\
Необхідно знайти таку точку $x_0\in L$, що:
\begin{equation*}
\nr{\overrightarrow{y-x_0}} \leq \nr{\overrightarrow{y-x}},\forall \vx\in L
\end{equation*}
Також необхідно опустити якимось чином перпендикуляр. Тобто:
\begin{equation*}
\overrightarrow{y-x_0} \perp \vx,\forall \vx\in L
\end{equation*}
Кожна ця задача має єдиний розв’язок і до того ж, вони однакові.
\end{teor}
\begin{proof}
$H = L \dotplus L^\perp$ - очевидний факт.\\
Це позначає, що $\forall \vy\in H: \exists! \vx_0\in L,\vx_1\in L^\perp \vy=\vx_0+\vx_1$\\
Очевидно, що це перпендикуляр. А оскільки розклад єдиний, то єдиність також гарантована.\\
$\nr{\vy-\vx}^2 = (\vy-\vx,\vy-\vx) = (\vy-\vx_0+\vx_0-\vx,\vy-\vx_0+\vx_0-\vx) = (\vy-\vx_0,\vy-\vx_0) + (\vx_0-\vx,\vx_0-\vx)+(\vy-\vx_0,\vx_0-\vx)+(\vx_0-\vx,\vy-\vx_0) = (\vy-\vx_0,\vy-\vx_0) + (\vx_0-\vx,\vx_0-\vx)$\\
Отже, в точці $\vx_0$ досягається мінімальна відстань. А єдиність знову гарантована.
\end{proof}
Використаємо теорему \ref{tr:3:1} підставивши в неї такі множини:\marginpar{\framebox{05.03.2014}}
\begin{eqnarray}
&H=\mL_2\cb{\Omg,\mC^n}\\
&L=\set{\tveta=C\vxi+\vd,C\in\mathcal{MAT}\cb{n\times m},\vd\in\mC^n}\\
&\forall \tveta, \cb{\veta-\hveta,\tveta} = 0 \label{tr:4:1}
\end{eqnarray}
З отриманої формули \eqref{tr:3:2} відомо, що:
\begin{equation}
\cb{\vxi,\veta} = \tr\mEt{\vxi\veta}
\end{equation}
Використаємо це для формули \eqref{tr:4:1}. Отже, $\forall C,\vd$ виконується:
\begin{eqnarray}
&\tr \mEt{\cb{\veta-\cb{A\vxi+\vb}}\cb{C\vxi+\vd}^\ast}=0\\
&\tr \mEt{\veta\vxi^\ast C^\ast-A\vxi\vxi^\ast C^\ast-\vb\vxi^\ast C^\ast + \veta\vd^\ast-A\vxi\vd^\ast - \vb\vd^\ast}=0
\end{eqnarray}
\begin{multline}
\tr\cb{\cb{C_{\vxi,\veta}+\vm_{\veta}\vm_{\vxi}^\ast}C^\ast - A\cb{\iDx+\vm_{\vxi}\vm_{\vxi}^\ast}C^\ast}  -{}\\{}- \tr\cb{b \vm_{\vxi}^\ast C^\ast + \vm_{\veta}\vd^\ast - A\vm_{\vxi}\vd^\ast -\vb\vd^\ast}=0
\end{multline}
\begin{eqnarray}
&\system{C_{\vxi,\veta}+\vm_{\veta}\vm_{\vxi}^\ast-A\cb{\iDx+\vm_{\vxi}\vm_{\vxi}^\ast}-\vb\vm_{\vxi}^*=0\\\vm_{\veta} - A\vm_{\vxi}-\vb=0}\\
&b=\vm_{\veta} - A\vm_{\vxi}\\
&C_{\veta,\vxi} - A\iDx=0 \Rightarrow A = C_{\veta,\vxi}\cdot D^{-1}_{\vxi}\\
&b=\vm_{\veta} - C_{\veta,\vxi}\cdot D^{-1}_{\vxi}\cdot\vm_{\vxi}
\end{eqnarray}
Отримали оцінку:
\begin{equation}
\hveta = C_{\vxi,\veta}\cdot D^{-1}_{\vxi} \vxi + \vm_{\veta} -  C_{\veta,\vxi}\cdot D^{-1}_{\vxi}\cdot\vm_{\vxi}
\end{equation}
Або, якщо спростити:
\begin{equation}
\hveta = \vm_{\veta} + C_{\vxi,\veta}D^{-1}_{\vxi}\cb{\vxi-\vm_{\veta}}
\end{equation}
\subsection{Дисперсійна матриця похибки}
Оскільки математичне сподівання оцінки нульове, виникає питання про її дисперсійну матрицю. Обчислимо дисперсійну матрицю похибки:
\begin{eqnarray}
&D_{\veta-\hveta} = \mEt{\cb{\veta-\hveta}\cb{\veta-\hveta}^\ast}\\
&D_{\hveta} = \mEt{\cb{ \vm_{\veta} + C_{\vxi,\veta}D^{-1}_{\vxi}\cb{\vxi-\vm_{\veta}}}\cb{ \vm_{\veta} + C_{\vxi,\veta}D^{-1}_{\vxi}\cb{\vxi-\vm_{\veta}}}^\ast}
\end{eqnarray}
%З цим потрібно щось придумати, але ідей в мене немає
\begin{multline}
D_{\veta-\hveta} = \mEt{\cb{\veta-\vm_{\veta}}\cb{\veta-\vm_{\veta}}^\ast} -{}\\{}- \mEt{\cb{\veta-\vm_{\veta}}\cb{C_{\vxi,\veta}D^{-1}_{\vxi}\cb{\vxi-\vm_{\veta}}^\ast}} - \mEt{C_{\vxi,\veta}\iDx^{-1}\cb{\vxi-\vm_{\vxi}}\cb{\veta-\vm_{\veta}}^\ast} +{}\\{}+ \mEt{C_{\vxi,\veta}\iDx^{-1}\cb{\vxi-\vm_{\vxi}}\cb{\vxi-\vm_{\vxi}}^\ast\cb{D^{-1}_{\vxi}}^\ast C^\ast_{\vxi,\veta}}
\end{multline}
\begin{equation}
D_{\veta-\hveta} = D_{\veta} - C_{\veta,\vxi}D^{-1}_{\vxi} C_{\vxi,\veta}
\end{equation}
\begin{multline}
\mEt{\nr{\veta-\hveta}^2} = \mEt{\mdl{\eta_1-\hat{\eta}_1}^2}+\ldots+\mEt{\mdl{\eta_n-\hat{\eta}_n}^2} ={}\\{}=\mDt{\eta_1-\hat{\eta}-1} + \ldots + \mDt{\eta_n + \hat{\eta}_n} = \tr\cb{D_{\eta}-C_{\veta,\vxi}D^{-1}_{\veta}C_{\vxi,\veta}}
\end{multline}
\section{Генератриси та їх застосування до випадкової кількості випадкових змінних. Гіллясті процеси Гальтона-Ватсона}
Нехай в нас є випадкова величина $\xi$.\\
\begin{tabular}{c|c|c|c|c|c}
$\xi$ & 0 & 1 & $\ldots$ & n & $\ldots$\\
\hline
$\mP$ & $p_0$ & $p_1$ & $\ldots$ & $p_n$ & $\ldots$
\end{tabular}\\
\textbf{Генератриса} випадкової величини $\xi$ це функція $G_\xi(z) = \suml_{k=0}^\infty p_k z^k$
\subsection{Властивості генератриси}
\begin{enumerate}
\item $G_\xi(1)=1$;
\item При $z\in\bb{0,1}$ ряд збігається рівномірно;
\item $G_\xi(z)\geq 0$ монотонно не спадна і опукла в широкому сенсі;
\item $G_\xi'(1)= \mEx$;
\item $G_\xi(z)=\mEt{z^\xi}$;
\item Якщо $\xi\perp\eta$, то $G_{\xi+\eta}=\mEt{z^{\xi+\eta}} = \mEt{z^\xi}\mEt{z^{\eta}} = G_\xi(z)\cdot G_\eta(z)$.
\end{enumerate}
Нехай є послідовність однаково розподілених величини $\xi_1,\ldots,\xi_n,\ldots\in\set{0,\ldots,1}$\\
$\nu$ є незалежна від них і також розподілена на $\set{0,\ldots,n,\ldots}$\\
Розглянемо $\Theta = \suml_{i=1}^\nu\xi_i$\\
\begin{multline}
G_{\Theta}(z) = \mEt{z^\Theta} = \mEt{\mnEt{z^\Theta}{\nu}} = \mEt{\mnEt{z^{\xi_1+\ldots+\xi_\nu}}{\nu}} ={}\\{}= \mEt{G^\nu_\xi(z)} = G_\nu\cb{G_\xi(z)}
\end{multline}
\begin{exs}
$\xi\sim Pois\cb{\la}$ - кількість студентів. \\
Студент здає іспит з ймовірністю $p$ і не здає з ймовірністю $1-p$.\\
Потрібно довести, що кількість тих, хто склад іспит також розподілена $\sim Pois\cb{\la p}$. \\
$\eta = \eps_1+\ldots+\epsilon_\xi$, де $\eps_i$ це одиниця, якщо студент іспит склав і нуль, якщо не склав.\\
$G_\xi(z) = e^{\la(z-1)}$\\
$G_\eps(z) = pz+1-p$\\
$G_\eta(z) = G_\xi(G_\eps(z)) = e^{\la\cb{pz+1+p-1}} = e^{\la p\cb{z-1}}$
\end{exs}
\subsection{Гіллясті процеси Гальтона-Ватсона}
\begin{description}
\item[Нульовий крок.] В нас є одна істота;
\item[Перший крок.] Ця істота народила $\xi^{(1)}$ нащадків і померла;
\item[Другий крок.] Кожна з цих істот-нащадків народжує ще нащадків $\xi^{(2)} = \suml_{i=1}^{\xi^{(1)}} \nu_i$, $\nu_i$ - незалежні і розподілені так само, як $\xi_1$.
\item[$\vdots$]
\end{description}
Введемо подію "Виродження" - для деякого кроку $n$ наше $\xi^{(n)}=0$.\\
Потрібно знайти ймовірність такої події.\\
Давайте позначимо $G_i(z)$ - генератриса кількості нащадків.\\
$G_1(z)$ - генератриса чисельності популяції на першому кроці. \\
$G_1(z) = G(z)$\\
$G_1(z) = G(G(z))$\\
$G_n(z) =  \underbrace{G(G(\ldots(z)\ldots)}_{n\text{ штук}}$\\
$\mP\set{\text{Виродження}} = \mP\set{\bcupl_{i=1}^\infty \text{на n-тому кроці нікого немає}}$\\
Ця послідовність подій, яка є неспадною. \\
$\liml_{n\to\infty}\mP\set{\text{на n-тому кроці нікого немає}} = \liml_{n\to\infty} G_n(0)=?$\\
Розглянемо деякі очевидні властивості цієї послідовності:
\begin{enumerate}
\item $\forall n\in\mN:G_n(0)\leq G_{n+1}(0)$;
\item $\forall n\in\mN:G_n(0)\leq 1$;
\end{enumerate}
Отже, за теоремою Вейерштраса існує границя.\\
Позначимо $\liml_{n\to\infty} G_n(0) = x$\\
\begin{eqnarray}
&G\cb{\liml_{n\to\infty} G_n(0)} = G(x)\\
&G\cb{\liml_{n\to\infty} G_n(0)} = \liml_{n\to\infty} G(G_n(0)) = \liml_{n\to\infty} G_{n+1}(0) = x
\end{eqnarray}
Отримали рівняння: $G(z) = z$
На жаль, в цього рівняння може бути кілька розв’язків і ми не будемо знайти, який саме нам потрібний. Тому нам необхідний засіб знайти кількість розв’язків. Зазначимо, що хоча б один розв’язок точно є, оскільки границя %вставити посилання
задовольняє рівняння та існує.\\
Можливі ситуації
%Тут були красиві малюнки, за бажанням можна намалювати
\begin{itemize}
\item Є лише один розв’язок;
\item Нескінченно багато розв’язків;
\item Два розв’язки.
\end{itemize}
Інші випадки неможливі.\\
%красивий малюнок
Генератриса має бути випуклою, а якщо розв’язків більше за два, з’являються точки перелому.\\
\begin{tver}
$\liml_{n\to\infty}G_n(0)=x$ - це найменший корінь рівняння $G(z)=z$ на проміжку $\bb{0,1}$.
\end{tver}
\begin{proof}
Нехай $y$ - це найменший розв’язок. Отже, $o\leq y$.\\
\begin{eqnarray*}
&G(0)\leq G(y)=y\\
&G_2(0)\leq G(y) = y\\
&\vdots
\end{eqnarray*}
Отже, $\liml_{n\to\infty} G_n(0)\leq y$.\\
Але ми знаємо, що ця границя також є розв’язком рівняння $G(z)=z$. \\
Таким чином, $\liml_{n\to\infty} G_n(0)=y$
\end{proof}
\begin{teor}
Ймовірність виродження гіллястого процесу Гальтона-Ватсона дорівнює найменшому невід’ємному розв’язку рівняння $G(z)=z$.
\end{teor}
\begin{exs}
\begin{tabular}{c|c|c}
0 & 1 & 2\\
\hline
$\frac12$ & 0 & $\frac12$
\end{tabular}
$G(z) = \cfrac{z^2+1}2$\\
$\cfrac{z^2+1}2 = z \Rightarrow z =1$\\
Отже, виродження з одиничною ймовірністю.
\end{exs}
\begin{exs}
\begin{tabular}{c|c|c}
0 & 1 & 3\\
\hline
$\frac12$ & 0 & $\frac12$
\end{tabular}
$G(z) = \cfrac{z^3+1}2$\\
$\cfrac{z^3+1}2 = z \Rightarrow z =\min\set{1,\cfrac{-1+\sqrt5}2} = \cfrac{-1+\sqrt5}2$\\
Отже, виродження з такою дивною ймовірністю.
\end{exs}
\marginpar{\framebox{12.03.2014}}
Коли ймовірність виродження дорівнює 1?\\
\begin{teor}
Якщо $\xi$ - кількість нащадків. Тоді ймовірність виродження дорівнює 1, якщо $\mEx\leq1$, і менше одинці, коли $\mEx>1$.
\end{teor}
\begin{nasl}[Виняток]
Якщо $\mP\set{\xi=1}=1$, то ймовірність виродження буде нульовою.
\end{nasl}
\begin{proof}
Нехай $G(z)\not\equiv z$\\
Доведемо, що:$\mEx>1\Rightarrow \mP_{d}<1$
\begin{equation}
\mEx = G'(1)
\end{equation}
В точці 1 - співпадають $G(1)=1$.\\
В околі 1 $G'(z)>z$\\
%Малюнок 1
Тоді $\exists z^*<1:G(z^*) = z^*$\\
Тепер, нехай $\mEx\leq 1$. Методом від супротивного $\mP_d<1$. Тоді $\exists z^*<1;G(z^*)=z^*$.\\
Тоді за теоремою Лагранжа $\exists y: G'(y)=1$. Але тоді похідна в останній точці $G'(1)>1$, а отже $\mEx>1$, а це протиріччя.
\end{proof}
\noindent За цим критерієм можна розбити процеси Гальтона-Ватсона на три типи:
\begin{enumerate}
\item $\mEx<1$ - \textbf{докритичний} випадок. Ймовірність виродження строго дорівнює 1;
\item $\mEx=1$ - \textbf{критичний} випадок. Ймовірність виродження строго дорівнює 1, окрім винятку;
\item $\mEx>1$ - \textbf{надкритичний} випадок. Ймовірність виродження строго менше за 1.
\end{enumerate}
\section{Ланцюги Маркова}
%малюнок 2
Розглянемо простір станів (або скінчений, або злічений) $E=\set{1,\ldots,n,(\ldots)}$. З’являються випадкові величини $\xi_k$ - номер стану, в якому знаходиться частинка після $k$-того кроку. $\xi_k\in E$.
\subsection{Марківська властивість}
Ця властивість є вкрай необхідною для ланцюгів Маркова.
\begin{equation}
\mP\set{\xi_{k+1}=i_{k+1}\setminus\set{\xi_1=i_1,\ldots,\xi_k=i_k}} = \mP\set{\xi_{k+1}=i_{k+1}\setminus\set{\xi_k=i_k}}
\end{equation}
Тобто, при теперішньому, що фіксовано, майбутнє не залежить від минулого.
Розглянемо таку ймовірність:
\begin{equation}
\mP\set{\xi_{k+1}=j\setminus \xi_k=i}
\end{equation}
Якщо ця ймовірність не залежить від $k$. Тобто, в будь-який момент часу ймовірність переходу зі стану $i$ в стан $j$ однакова, то це ланцюг Маркова називають \textbf{однорідним ланцюгом Маркова} і саме з такими однорідними ланцюгами Маркова ми будемо працювати.\\
Оскільки ланцюг Маркова у нас тепер однорідний, можемо ввести таке позначення:
\begin{equation}
\mP\set{\xi_{k+1}=j\setminus \xi_k=i} = \pij
\end{equation}
Надалі, деякий час припускаємо, що $E$ - скінченний простір аж до спеціального попередження. В такому випадку ми можемо розглянути матрицю:
\begin{equation}
P = \begin{pmatrix}
p_{11} & \ldots &p_{1n}\\
\vdots & \vdots & \vdots\\
p_{n1} & \ldots & p_{nn} 
\end{pmatrix}
\end{equation}
$P$ - це \textbf{матриця перехідних ймовірностей}.
\subsection{Властивості матриці перехідних ймовірностей}
\begin{enumerate}
\item $\forall i,j\in\set{1,\ldots,n}:\pij\in\bb{0,1}$;
\item $\suml_{j=1}^n \pij = 1,\forall i\in\set{1,\ldots,n}$, тобто $P$ відноситься до класу \textbf{стохастичних матриць};
\item $P-I$ - матриця, у якої $\suml_{j=1}^n \pij-1 = 0,\forall i\in\set{1,\ldots,n}$; Отже, $\det\cb{P-I}=0$. А з цього випливає, що $1$ - це власне число матриці $P$;
\end{enumerate}
\subsection{Дві основні характеристики ланцюга Маркова}
Це матриця перехідних ймовірностей $P$ та початковий розподіл (який задається у вигляді ковектора) $\kvp^{(0)}$\\
Знайдемо ймовірність на $k+1$-шому кроці.
\begin{multline}
p_i^{(k+1)} = \mP\set{\xi_{k+1} = i} = \suml_{j=1}^n\mP\set{\xi_k = j}\mP\set{\xi_{k+1} = i \setminus  \xi_k=j} ={} \\ {} = \sumjon p_j^{(k)} p_{ji} = p_1^{(k)} p_{1i}+\ldots+p_n^{(k)}p_{ni}
\end{multline}
Отже, тепер ми можемо стверджувати, що 
\begin{eqnarray}
\kvp^{(1)} &=& \kvp^{(0)} \cdot P\\
\kvp^{(2)} &=& \kvp^{(1)} \cdot P = \kvp^{(0)}\cdot P^2\\
&\vdots&\nonumber\\
\kvp^{(k+1)} &=& \kvp^{(k)} \cdot P =\kvp^{(0)}\cdot P^k
\end{eqnarray}
\begin{exs}
Є дядя Гриша. У нього є три стани: дім, завод і пивна. Задаємо матрицю переходів:
%Розмір точок
\begin{tikzpicture}[node distance=3cm]
\node[circle,fill,label=left:Дім=1cm] (c1) {};
\node[circle,fill,below right=of c1,label=below:Пивна] (c3) {};
\node[circle,fill,above right=of c3,label=right:Завод] (c2) {};
\path[->]
(c1) edge[bend left] node[above]{$\frac13$} (c2)
(c2) edge node[below right]{$\frac13$} (c1)
(c1) edge[bend right] node[below]{$\frac23$} (c3)
(c3) edge node[above right]{$\frac23$} (c1)
(c2) edge[bend left] node[below]{$\frac23$} (c3)
(c3) edge node[above]{$\frac13$} (c2)
;
\end{tikzpicture}\\
Задача. Відомо, що дядя Гриша був вдома. Ймовірність того, що після $k$-го кроку він буде в пивній.
\begin{eqnarray*}
&\kvp^{(0)} = \vect{1\\0\\0}\\
&P = \begin{pmatrix}
0 & \frac13 & \frac23 \\
\frac13 & 0 & \frac23 \\
\frac23 & \frac13 & 0
\end{pmatrix}
\end{eqnarray*}
Шукаємо власні числа цієї матриці. Одне з цих власних чисел, це $1$.\\
\begin{equation*}
\det \cb{A-\la I} = \begin{vmatrix}
-\la & \frac13 & \frac23 \\
\frac13 & -\la & \frac23 \\
\frac23 & \frac13 & -\la
\end{vmatrix} = -\la^3 +\cfrac79\la + \cfrac29
\end{equation*}
Поділимо цей многочлен на $\la-1$: $-\la^2-\la-\cfrac29$\\
Після розв’язку цього рівняння отримуємо три власні числа: $\la=\set{1,-\cfrac23,-\cfrac13}$\\
Отже, після переходу отримаємо таку матрицю:
\begin{equation*}
P_i = \begin{pmatrix}
1 & 0 & 0\\
0 & -\cfrac13 & 0 \\
0 & 0 & -\cfrac23
\end{pmatrix}
\end{equation*}
Перемножимо її $k$ разів, щоб отримати правильну відповідь:
\begin{equation*}
P_i^k = \begin{pmatrix}
1 & 0 & 0\\
0 & (-1)^k\cfrac1{3^k} & 0 \\
0 & 0 & (-1)^k\cfrac{2^k}{3^k}
\end{pmatrix}
\end{equation*}
Спробуємо зхитрити та не знаходити власні вектори: $p_3^{(k)}$ - лінійне перетворення $\vect{1\\ (-1)^k\cfrac1{3^k}\\(-1)^k\cfrac{2^k}{3^k}}$.\\
Отже, отримуємо, що:
\begin{equation*}
p_3^{(k)} = a+ (-1)^k\cfrac1{3^k}b +(-1)^k\cfrac{2^k}{3^k}c
\end{equation*}
Підставимо різні значення $k$, щоб створити систему: 
\begin{eqnarray}
p_3^{(0)} &=& 0\\
p_3^{(1)} &=& \cfrac23\\
p_3^{(2)} &=& \cfrac29
\end{eqnarray}
Отже, отримали систему:
\begin{eqnarray}
0 &=& a + b + c\\
\cfrac23 &=& a - \cfrac b3 - \cfrac{2c}3\\
\cfrac29 &=& a + \cfrac b9 + \cfrac49 c
\end{eqnarray}
Розв’яжемо цю систему:
\begin{eqnarray}
a&=& \cfrac25\\
b&=& 0\\
c&=& -\cfrac25\\
\end{eqnarray}
Отже, отримали відповідь:
\begin{equation*}
p_3^{(k)} = \cfrac25  +(-1)^{k+1}\cfrac{2^{k+1}}{5\cdot 3^k}
\end{equation*}
\end{exs}
\marginpar{\framebox{19.03.2014}}
Чи обов’язково існує розподіл $\liml_{n\to\infty} \kvp^{(n)}$?\\
Як цей розподіл пов’язаний з початковими?\\
\begin{exs}
%малюнок
\begin{eqnarray*}
\kvp^{(0)} &=& \cb{1,0}\\
\kvp^{(2k)} &=& \cb{1,0}\\
\kvp^{(2k+1)} &=& \cb{0,1}
\end{eqnarray*}
\end{exs}
\begin{exs}
%малюнок 2
\begin{eqnarray}
\kvp^{(0)} &=& \cb{p,q}\\
\kvp^{(2k)} &=& \cb{p,q}\\
\kvp^{(2k+1)} &=& \cb{p,q}
\end{eqnarray}	
\end{exs}
\begin{teor}[Ергодична теорема Маркова]
Нехай ОЛМ і $\exists k\in\mN: \min\limits_{i,j=1,\ldots,n} \pij^{k}>0$\\
Тобто, існує таке $k$, що за $k$ кроків можна перейти звідки завгодно куди завгодно.\\
Тоді 
\begin{itemize}
\item $\exists \liml_{r\to\infty} \mP^{r} = \mP^{\infty} = \Pi$;
\item $\Pi = \begin{pmatrix}
\Pi_1 & \ldots & \Pi_n \\
\vdots & \vdots & \vdots \\
\Pi_1 & \ldots &\Pi_n
\end{pmatrix}$;
\item $\kv\Pi$ може бути знайдений, як єдиний розв’язок СЛАР
\begin{equation*}
\system{&\kv\Pi\mP=\kv\Pi \\ &\Pi_1+\ldots+\Pi_n=1}
\end{equation*}
\end{itemize}
\end{teor}
\begin{proof}
Розглянемо матрицю переходу за один крок
\begin{equation}
\mP = \begin{pmatrix}
p_{11} & \ldots & p_{1n} \\
\vdots & \vdots & \vdots\\
p_{n1} & \ldots &p_{nn}
\end{pmatrix}
\end{equation}
\begin{equation}
m_j^{(n)} = \min\limits_i \pij^{(n)}
\end{equation}
\begin{equation}
M_j^{(n)} = \max\limits_i\pij^{(n)}
\end{equation}
\begin{tver}
\begin{eqnarray}
m_j^{(r+1)} &\geq& m_j^{(r)}\\
M_j^{(r+1)} &\leq& M_j^{(r)}
\end{eqnarray}
\begin{eqnarray}
&m_j^{(r+1)} = \min\limits_i \pij^{(r+1)}\\
&\pij^{(r+1)} = \suml_{\al=1}^n p_{i\al}^{(1)}p_{\al j}^{(r)} \geq \suml_{\al=1}^n p_{i\al}^{(1)} \cdot m_j^{(r)} = m_j^{(r)} \suml_{\al=1}^n p_{i\al}^{(1)} = m_j^{(r)}\\
&\Rightarrow m_j^{(r+1)} \geq m_j^{(r)}
\end{eqnarray}
\end{tver}
Залишилося довести, що $M_j^{(r_l)} - m_j^{(r_l)}\to 0$\\
\begin{equation}
M_j^{(k+r)} = \max\limits_i \pij^{(k+r)}
\end{equation}
\begin{equation}
\min\limits_i \pij^{(k)} = \eps
\end{equation}
\begin{multline}
\pij^{(k+r)} = \suml_{\al=1}^n p_{i\al}^{(k)} p_{\al j}^{(r)} = \suml_{\al=1}^n \underbrace{\cb{p_{i\al}^{(k)} -p_{j\al}^{(r)}\eps}}_{\geq 0} p_{\al j}^{(r)} + \suml_{\al=1}^n p_{j\al}^{(r)} \eps p_{\al j}^{(r)} \leq \suml_{\al=1}^n \cb{p_{i\al}^{(k)} - p_{j\al}^{(r)}\eps}M_j^{(r)} + \eps p_{jj}^{(2r)} = {} \\ {} = M_j^{(r)} \cb{\suml_{\al=1}^n \cb{p_{i\al}^{(k)} - p_{j\al}^{(r)}\eps}} + \eps p_{jj}^{(2r)} = (1-\eps) M_j^{(r)} +\eps p_{jj}^{(2r)}
\end{multline}
\begin{equation}
\Rightarrow M_j^{(k+r)} \geq (1-\eps) M_j^{(r)} +\eps p_{jj}^{(2r)}
\end{equation}
Аналогічно можна отримувати з мінімумом:
\begin{equation}
\Rightarrow m_j^{(k+r)} \leq (1-\eps) m_j^{(r)} +\eps p_{jj}^{(2r)}
\end{equation}
Остаточно ми отримуємо:
\begin{equation}
M_j^{(k+r)} - m_j^{(k+r)} \leq (1-\eps)\cb{M_j^{(r)}-m_j^{(r)}}
\end{equation}
\begin{eqnarray}
r=0:& M_j^{(k)} - m_j^{(k)} \leq (1-\eps)\cb{M_j^{(0)}-m_j^{(0)}}\\
&M_j^{(2k)} - m_j^{(2k)} \leq (1-\eps)^2\cb{M_j^{(0)}-m_j^{(0)}}\\
&\vdots\nonumber\\
&M_j^{(l\cdot k)} - m_j^{(l\cdot k)} \leq (1-\eps)^l\cb{M_j^{(0)}-m_j^{(0)}}\\
\end{eqnarray}
Отже, отримали:
\begin{equation}
l\to\infty M_j^{(lk)} - m_j^{(lk)} \to 0
\end{equation}
Отже, за підпослідовністю у нас є збіжність. А після цього і для всієї послідовності.\\
Як знайти $\Pi_1,\ldots,\Pi_n$?\\
\begin{equation}
\Pi = \liml_{r\to\infty} \mP^{r}
\end{equation}
$\mP^r$ - матриця переходу за $r$ кроків. Сума в кожному рядку дорівнює одиниці. \\
Тоді і для матриці $\Pi$ сума в рядку дорівнює одинці.\\
\begin{equation}
\suml_{i=1}^n \Pi_i = 1
\end{equation}
\begin{eqnarray}
&\mP^n\mP = \mP^{n+1}\\
n\to\infty:&\Pi\mP = \Pi\\
&\begin{pmatrix}
\Pi_1 & \ldots & \Pi_n \\
\vdots & \vdots & \vdots \\
\Pi_1 & \ldots &\Pi_n
\end{pmatrix} \mP = \begin{pmatrix}
\Pi_1 & \ldots & \Pi_n \\
\vdots & \vdots & \vdots \\
\Pi_1 & \ldots &\Pi_n
\end{pmatrix} \\
& \kv\Pi\mP=\kv\Pi
\end{eqnarray}
Отже, $\Pi$ можна знайти, як розв’язок системи 
\begin{equation}
\system{\kv\Pi\mP=\kv\Pi\\ \suml_{i=1}^n \Pi_i=1}
\end{equation}
Залишилося довести, що ця система має лише єдиний розв’язок.\\
Якщо $\kvp^{(0)}$, то який буде $\kvp^{(\infty)}$
\begin{multline}
\kvp^{(\infty)} = \liml_{n\to\infty} \kvp^{(n)} = \liml_{n\to\infty} \kvp^{(0)} \cdot \mP^n = \kvp^{(0)} \liml_{n\to\infty} \mP^n = \kvp^{(0)}\Pi ={} \\ {} = \cb{p_1^{(0)},\ldots,p_n^{(0)}} \cdot \begin{pmatrix}
\Pi_1 & \ldots & \Pi_n \\
\vdots & \vdots & \vdots \\
\Pi_1 & \ldots &\Pi_n
\end{pmatrix} = \cb{\Pi_1,\ldots,\Pi_n}
\end{multline}
Отже, для кожного розподілу граничним розподілом є $\Pi$.\\
Нехай, окрім розв’язку $\kv\Pi$ є ще розв’язок $\vec\nu$\\
\begin{eqnarray}
&\vec\nu\mP^{\infty} = \vec\nu \Pi = \kv\Pi\\
&\vec\nu\mP^{\infty} = \liml_{n\to\infty} \vec\nu \mP^n = \vec\nu
\end{eqnarray}
\end{proof}
\begin{exs}
Згадаємо дядю Гришу. Його матриця переходів:
\begin{equation}
P = \begin{pmatrix}
0 & \frac13 & \frac23 \\
\frac13 & 0 & \frac23 \\
\frac23 & \frac13 & 0
\end{pmatrix}
\end{equation}
Хочемо знайти розподіл на нескінченності. Перевіримо спочатку умови за графом:
\begin{tikzpicture}[node distance=3cm]
\node[circle,fill,label=left:Дім=1cm] (c1) {};
\node[circle,fill,below right=of c1,label=below:Пивна] (c3) {};
\node[circle,fill,above right=of c3,label=right:Завод] (c2) {};
\path[->]
(c1) edge[bend left] node[above]{$\frac13$} (c2)
(c2) edge node[below right]{$\frac13$} (c1)
(c1) edge[bend right] node[below]{$\frac23$} (c3)
(c3) edge node[above right]{$\frac23$} (c1)
(c2) edge[bend left] node[below]{$\frac23$} (c3)
(c3) edge node[above]{$\frac13$} (c2)
;
\end{tikzpicture}\\
При $k=2$ умови виконані. Отже, можемо записати систему:
\begin{equation}
\system{\kv\Pi\mP=\kv\Pi \\ \Pi_1+\Pi_2+\Pi_3 = 1}
\end{equation}
\begin{equation}
\system{\cfrac13\Pi_2+\cfrac23\Pi_3 = \Pi_1 \\
\cfrac13\Pi_1+\cfrac13\Pi_3 = \Pi_2\\
\cfrac23\Pi_1 + \cfrac23\Pi_2 = \Pi_3\\
\Pi_1+\Pi_2+\Pi_3 = 1
}
\end{equation}
Отримали розв’язок:
\begin{equation}
\kv\Pi=\cb{\cfrac7{20},\cfrac14,\cfrac25}
\end{equation}
\end{exs}
\subsection{Частота потрапляння в деякий стан}
\begin{equation}
\nu_k^{(r)} = \cfrac{\mI_{\xi_1=k}+\ldots+\mI_{\xi_r=k}}{r}
\end{equation}
\begin{teor}[Закон великих чисел для ОЛМ]
Якщо виконується умова ергодичності, тобто $\min\limits_{i,j}p_{ij}^{(k)} >0,\exists k\in\mN$\\
То $\exists \mP\liml_{r\to\infty} \nu_k^{(r)} = \Pi_k$\\
\end{teor}
\begin{proof}
Ідея доведення: будемо доводити збіжність в середньому квадратичному. За критерієм, потрібно перевірити, що:
\begin{eqnarray}
\mEt{\cfrac{\mI_{\xi_1=k}+\ldots+\mI_{\xi_r=k}}{r}} &\to&\Pi_k\\
\mDt{\cfrac{\mI_{\xi_1=k}+\ldots+\mI_{\xi_r=k}}{r}} &\to&0
\end{eqnarray}
Розберемося спочатку з першим фактом:
\begin{multline}
\mEt{\cfrac{\mI_{\xi_1=k}+\ldots+\mI_{\xi_r=k}}{r}} \to\Pi_k = \cfrac1r \cb{\mP\set{\xi_1=k}+\ldots+\mP\set{\xi_r=k}} ={} \\ {}= \cfrac1r \cb{p_k^{(1)} +\ldots + p_k^{(r)}} \xrightarrow[r\to\infty]{}\Pi_k
\end{multline}
\end{proof}
\section{Знаходження ймовірності та середніх часів досягнення множини станів} \marginpar{\framebox{26.03.2014}}
Починаючи з цього моменту $E$ може бути ліченим\\
$A\subset E$, $A$ - містить деякі стани\\
$\tau$ - момент першого потрапляння в $A:\tau = \min\set{n\geq 0:\xi_n\in A}$ \\
$f_i$ - ймовірність $\set{\tau<\infty\diagup_{\xi_0=i}}$ - ймовірність хоч колись потрапити в множину станів $A$, якщо процес починається з $i$-того стану.\\
Як знаходити $f_i$?:
\begin{equation}\label{tr:6:1}
	\system{f_i=1,\quad i\in A\\ f_i = \suml_{j\in E} p_{ij} \cdot f_j, i\not\in A}
\end{equation}
\begin{teor}
$\cb{f_i}i\in E$ - є найменшим невід’ємним роз’язком системи \eqref{tr:6:1}
\end{teor}
\begin{proof}
Якщо $\cb{f_i}i\in E$ - справжні ймовірності потрапляння в $A$, то $\cb{f_i}i\in E$ задовольняє системі \eqref{tr:6:1}\\
Навпаки: Нехай $\cb{f_i}i\in E$ - розв’язок (невід’ємний, деякий)
\begin{itemize}
	\item Якщо $i\in A$: Тоді у системі $f_i=1$. З іншого боку $P\cb{A} = 1$
	\item Якщо $i\not\in A$: $f_i = \suml_{j\in A}p_{ij} f_j + \suml_{j\not\in A}p_{ij}f_j = \suml_{j\in A}p_{ij} + \suml_{j\not\in A}p_{ij}f_j = \suml_{j\in A} + \suml_{j\not\in A}\suml_{k\in E} p_{ij}p_{jk}f_k = \suml_{j\in A} p_{ij} + \suml{j\not\in A}\suml_{k\in A} p_{ij}p_{jk}\cdot 1 + \suml_{j\not\in A} \suml_{\not\in A} p_{ij}p_{jk}f_k = ...$\\
	В решті решт отримуємо:
	$
		\suml_{i_1\in A}p_{ii_1} + \suml_{i_1\not\in A}{i_2\in A} p_{ii_1}p_{ii_2} + \ldots + \suml_{i_1\not\in A} \suml_{i_2\not\in A} \ldots\suml_{i_n\in A} p_{ii_1}\ldots p_{i_{n-1}i_n} + \suml_{i_1\not\in A} \suml_{i_2\not\in A} \ldots\suml_{i_n\not\in A} p_{ii_1}\ldots p_{i_{n-1}i_n} f_n\geq 0
	$\\
	Оскільки $\suml_{i_1\in A}p_{ii_1} + \suml_{i_1\not\in A}{i_2\in A} p_{ii_1}p_{ii_2} + \ldots + \suml_{i_1\not\in A} \suml_{i_2\not\in A} \ldots\suml_{i_n\in A} p_{ii_1}\ldots p_{i_{n-1}i_n} = \mP\set{\tau=i\diagup_{\xi_0=i}} + \ldots + \mP\set{\tau=n\diagup_{\xi_0=i}}$\\
	Отримуємо, що $f_i \geq \mP\set{t\leq n\diagup_{\xi_0=i}}$.\\
	Якщо $n\to\infty$, то отримуємо $f_i\geq \mP\set{\tau<\infty\diagup_{\xi_0=i}}$. Отже, потрібно взяти найменший невід’ємний розв’язок.
\end{itemize}
\end{proof}
\begin{exs}
Дискретний процес розмноження та загибелі.\\
$\xi_0,\xi_1,\ldots$\\
Початкові дані $\xi_0=k$. Розвиток: $\xi_{n+1} = \xi_n \pm 1$ з ймовірністю $p$ та $q$ відповідно. \\
Нехай $k>0$. Яка ймовірність того, що популяція виродиться?\\
$A=\set{k=0} = \set{0},f_k = \mP\set{\tau<\infty\diagup_{\xi_0=k}}$\\
Складаємо систему
\begin{equation}
	\system{f_0=1\\ f_k = pf_{k+1} + qf_{k-1},\forall k\geq 1}
\end{equation}
Якщо корені $\la_1$ та $\la_2$ характеристичного рівняння дійсні та різні, то $f_k = C_1 \la_1^k + C_2\la_2^k$\\
Якщо $\la_1=\la_2=\la$, то $f_k=C_1 \la^k + C_2\la^k$\\
$\la=p\la^2 + q\quad\Rightarrow\quad \la_1 = 1;\la_2=\cfrac qp$\\
Нехай $q\neq p$, тоді $f_k = C_11^k + C_2 \cb{\cfrac qp}^k$ або $f_k = C_1 + C_2\cb{\cfrac qp}^k$\\
$1=f_0 = C_1 + C_2 \Rightarrow C_1 = 1 - C_2$\\
Отримуємо $f_k = 1 - C_2 + C_2\cb{\cfrac qp}^k$ або \framebox{$f_k = 1 - \cb{\cb{\cfrac qp}^k - 1}C_2$}\\
\begin{description}
\item[Якщо p<q] $\cb{\cfrac qp}^k -1\xrightarrow[k\to\infty]{} \infty \Rightarrow C_2\geq 0$\\
	$\cb{\cfrac qp}^k -1\geq 0\min\Rightarrow C_2 = 0$, отже $f_k=1$
\item[Якщо p>q] $\cb{\cfrac qp}^k -1\xrightarrow[k\to\infty]{} -1 \Rightarrow C_2\leq $\\
	$\cb{\cfrac qp}^k -1<1\max\Rightarrow C_2 = 1$, отже $f_k=\cb{\cfrac qp}^k$
\item[Якщо p=q] $f_k=1$
\end{description}
\end{exs}
Як знайти $m_k = \mEt{\tau\diagup_{\xi_0=k}}$\\
Має сенс шукати, якщо $\forall k: f_k=1$ за допомогою такої системи:
\begin{equation}\label{tr:6:2}
	\system{m_k=0,\quad k\in A\\ m_k = \suml_{l\in E} p_{kl} m_l + 1,k\not\in A}
\end{equation}
\begin{teor}
Середні кількості кроків до $A$ є найменшим невід’ємним розв’язком системи \eqref{tr:6:2}
\end{teor}
\begin{proof}
Якщо є середні кількості, то вони задовольняють \eqref{tr:6:2}\\
Навпаки: нехай $(m_k),k\in E$ - деякий розв’язок (невід’ємний).
\begin{equation*}
k\in A:\quad m_k=0\\
\end{equation*}
\begin{multline*}
k\not\in A:\quad m_k = 1 + \suml_{k_1\in E} p_{kk_1} m_{k_1} = 1 + \suml_{k_1\in A} p_{kk_1} m_{k_1} + \suml_{k_1\not\in A}p_{kk_1} m_{k_1}= \ldots {}\\{}\ldots= 1 + \suml_{k_1\in A} p_{kk_1} + \ldots + \suml_{k_1\not\in A} \ldots\suml_{k_n\in A} p_{kk_1}\ldots p_{k_{n-1}k_n} +{}\\{}+ \suml_{k_1\not\in A} \ldots\suml_{k_n\not\in A} p_{kk_1}\ldots p_{k_{n-1}k_n} m_{k_n} 
\end{multline*}
Отримали $m_k \geq 1 + \suml_{k_1\in A} p_{kk_1} + \ldots + \suml_{k_1\not\in A} \ldots\suml_{k_n\in A} p_{kk_1}\ldots p_{k_{n-1}k_n}$\\
Або $m_k \geq \mP\set{\tau\geq 1\diagup_{\xi_0=k}} + \ldots + \mP\set{\tau\geq n+1\diagup_{\xi_0=k}}$
\begin{war}
$\mP\set{\tau\geq 1} + \ldots  = \cb{p_1+ p_2 + p_3 + \ldots} + \cb{p_2 + p_3 +\ldots} +\ldots = p_1 + 2p_2 + 3p_3 + \ldots = \mEt{\tau}$
\end{war}
\noindent Звісно, при $n\to\infty$ отримуємо $m_k \geq \suml_{l=1}^\infty \mP\set{\tau\geq l\diagup_{\xi_0=k}} = \mEt{\tau\diagup_{\xi_0=k}}$ - справжня середня кількість кроків до $A$. Тому потрібно брати найменший невід’ємний розв’язок.
\end{proof}
\begin{exs}
Дискретний розподіл розмноження та загибелі. $p\leq q$, $f_k=1$.\\
Побудуємо систему:
\begin{equation}
	\system{m_0 =0 \\ m_k = 1 + pm_{k+1} + qm_{k-1}}
\end{equation}
\begin{equation}
	\system{m_2 = m_{2\to 1} + m_{1\to 0} = 2m_1 \\ m_k = km_1 = C_k}
\end{equation}
$C_k = 1 + p C(k+1) + q C(k-1) = 1 + C_k + pC - qC \Rightarrow C = \cfrac1{q-p}$\\
Отже, отримали:
\begin{equation}
	m_k = \system{\cfrac k{q-p}, q>p\\ \infty , q=p}
\end{equation}
\end{exs}



\marginpar{\framebox{02.04.2014}}
\begin{teor}
$\forall e\in E:e$ - або рекурентний, або транзієнтний.
\end{teor} 

\chapter{Випадкові процеси} 
\section{Основні поняття}\marginpar{\framebox{09.04.2014}}
\textbf{Випадковий процес} $\cb{X(t),t\in T}$ - це сукупність випадкових величин $X(t)$, індексованих параметром $t\in T$, які задані на спільному ймовірнісному просторі $\cb{\Omg,\iF,\mP}$\\
Розглянемо деякі випадки:
\begin{itemize}
\item $T={1}$ - випадкова величина;
\item $T={1,2,\ldots,n}$ - випадковий вектор;
\item $T=\mN$ - випадкова послідовність. Також, цей випадок допускає будь-яку злічену множину;
\item $T=\bb{a,b},\cb{a,b},\ldots$ - тобто, деякий інтервал. І це буде називатися просто випадковий процес.
\item $T=\mR^d$ - випадкове поле.
\end{itemize}
Фактично, параметр $t$ можна ототожнювати з часом.\\
Отже, випадковий процес $X$ - це функція від двох змінних $X\cb{t,\omg}$. Тоді, з цієї точки зору, можна сказати, що $X:T\times\Omg\to \mR$. Оскільки нам необхідна випадковість, то необхідно вимагати, щоб:
\begin{equation}
\forall t\in T:X(t,\cdot) - \text{вимірна, відсно }\iF\setminus \iB\cb{\mR}
\end{equation}
$X(t,\cdot)$ - випадкове значення нашого процесу $X$ в момент $t$.\\
$X(\cdot,\omg)$ - \textbf{траєкторія} або \textbf{реалізація} випадкового процесу $X$.\\
\begin{exs}
В нас є параметрична множина $T=\cb{0,90}$.\\
В момент $\tau$ на пару приходить Павло. $\tau\sim U\cb{0,90}$.\\
Тепер введемо деякий випадковий процес: $X(t)$ - кількість Павлів на парі у момент $t$.\\
\begin{equation*}
X(t)=\system{
0, t<\tau\\
1, t\geq \tau
}
\end{equation*}
%Намалюємо траєкторію:\\
%Намалювати графік цієї штуки
\end{exs}
\subsection{Основна характеристика випадкового процесу}
Основна характеристика випадкового процесу це \textbf{система скінченновимірних розподілів}.
\begin{itemize}
\item Одновимірні функції розподілу:\\
\begin{equation}
F_X(t,x) = \mP\set{X(t)<x}
\end{equation}
Одновимірний розподіл:
\begin{equation}
P_t(B) = \mP\set{X(t)\in B};B\in\iB\cb{\mR^1}
\end{equation}
\item Двовимірні функції розподілу:\\
\begin{equation}
F_X(t_1,t_2,x_1,x_2) = \mP\set{X(t_1)< x_1 \cap X(t_2) < x_2}
\end{equation}
Двовимірний розподіл:
\begin{equation}
P_{t_1,t_2} (B) = \mP\set{(X(t_1),X(t_2)}\in B;B\in\iB\cb{\mR^n}
\end{equation}
І так далі можна ввести $n$-вимірні функції розподілу і $n$-вимірний розподіл.
\end{itemize}
\subsubsection{Система випадкових розподілів}
Потрібно задати усю систему - $\forall n\geq 1,\forall t_1,\ldots,t_n \in T$.
\begin{exs}
\begin{equation*}
X(t) = \system{0,t<\tau\\1,t\geq \tau}
\end{equation*}
$t$ - фіксуємо.\\
\begin{center}
\begin{tabular}{|c|c|c|}
$X(t)$ & 0 & 1 \\
\hline
$\mP$ & $1-\cfrac t{90}$ & $\cfrac t{90}$
\end{tabular}
\end{center}
Двовимірний розподіл: фіксуємо $t_1,t_2(t1\leq t_2)$.\\
\begin{center}
\begin{tabular}{|c|c|c|}
$X(t_2)\setminus X(t_1)$ & 0 & 1 \\
\hline
0 & $1 - \cfrac t{90}$ & 0 \\
\hline
1 & $\cfrac{t_2-t_1}{90}$ & $\cfrac{t_1}{90}$
\end{tabular}
\end{center}
$X(t_1) = X(t_2) =0$\\
\end{exs}
\subsection{Властивості скінченновимірного розподілу}
Розглянемо $P_{t_1,\ldots,t_n} (B),B\in\iB\cb{\mR^n}$.\\
%Знайти у когось тут властивосты
\begin{teor}[Теорема Колмагорова]
Нехай задана система скінченновимірних розподілів: $P_{t_1,\ldots,t_n}(B_1\times \ldots\times B_n),\forall n\in\mN$:\\
яка задовольняє попереднім умовам %тим, які потрібно у когось знайти
Тоді, на деякому ймовірнісному просторі $\cb{\Omg,\iF,\mP}$ існує випадковий процес $X(t,\omg)$, який має саме цю систему скінченних розподілів.
\end{teor}
\section{$\mL_2$-процеси}
$X$ - це \textbf{$\mL_2$-процес}, якщо:
\begin{equation}
\forall t:\mEt{\mdl{X(t)}}<\infty \vee \mEt{\mdl{X(t)}^2}<\infty
\end{equation}
Тоді можемо ввести \textbf{математичне сподівання}:
\begin{equation}
m(t) = \mEt{X(t)},t\in T
\end{equation}
Це може бути будь-яка функція, без будь-яких обмежень.\\
\textbf{Кореляційна функція}:
\begin{equation}
C_x(s,t) = \mEt{\cb{X(s)-m(s)}\overline{\cb{X(t)-m(t)}}} = \mEt{X(s)\overline{X(t)}} - \cb{\mEt{X(s)}}\overline{\cb{\mEt{X(t)}}}
\end{equation}
Чому вона існує:
\begin{equation}
\mdl{\mEt{\cb{X(s)-m(s)}\overline{\cb{X(t)-m(t)}}}} \leq \sqrt{\mEt{\mdl{X(s)-m(s)}^2} + \mEt{\mdl{X(t)-m(t)}}^2} < \infty
\end{equation}
\subsection{Властивості кореляційної функції}
\begin{teor}[Ермітовість]
$C_x(s,t) = \overline{C_x(t,s)}$ 
\end{teor}
\begin{proof}
Позначимо: $X_0(t) = X(t) - m(t)$\\
\begin{equation*}
C_x(t,s) = \mEt{X_0(t)\overline{X_0(s)}} = \overline{\mEt{X_0(s)\overline{X_0(t)}}} = \overline{C_x(s,t)}
\end{equation*}
\end{proof}
\begin{teor}
Розглянемо $\vec X = \vect{X(t_1)\\ \vdots \\ X(t_n)}$\\
$C_{\vec X} = \mdl{\mdl{C_x(t_i,t_j)}}_{i,j=1,\ldots,n}$ \\
Оскільки це дуже схоже на кореляційну матрицю, вона отримує деяку її властивість:
$\forall t_1,\ldots,t_n:\mdl{\mdl{C_x(t_i,t_j)}}_{i,j=1,\ldots,n}$ - невід’ємно визначена.
\end{teor}
А це означає, що така квадратична форма:
\begin{equation}
\cb{\mdl{\mdl{C_x(t_i,t_j)}}_{i,j=1,\ldots,n} \vec b,\vec b} \geq 0,\forall \vec b \in \mC^n
\end{equation}
Або, якщо переписати у вигляді суми:
\begin{equation} \label{tr:8:1}
\suml_{i=1}^n \suml_{j=1}^n C_x\cb{t_i,t_j} b_i\overline{b_j} \geq 0,\forall b_1,\ldots,b_n\in\mC
\end{equation}
Отже, кореляційна функція завжди ермітово симетрична і задовольняє властивість \eqref{tr:8:1}. Функції, що задовольняють умові \eqref{tr:8:1}, то вони називаються \textbf{невід’ємно визначеними}.
\begin{teor}
Якщо функція є ермітово симетричною та невідємно визначеною, то вона є кореляційною функцією.
\end{teor}
\begin{proof}
Фіксуємо $n\geq 1$.\\
Фіксуємо $t_1,\ldots,t_n\in T$.\\
Тоді матриця $С = \mdl{\mdl{C_x(t_i,t_j)}}_{i,j=1,\ldots,n} $ - є невід’ємно визначеною та ермітовою.\\
Тоді, $P_{t_1,\ldots,t_n} (B)=$ розподіл Гаусовського вектора з кореляційною матрицею $\mdl{\mdl{C_x(t_i,t_j)}}_{i,j=1,\ldots,n}$\\
Тоді, за теоремою Колмагорова, можна побудувати процес.
\end{proof}
\begin{exs}
$T = \bb{0,+\infty}$\\
$C(s,t) = \min\set{s,t}$\\
\begin{itemize}
\item $\min{s,t} = \min\set{t,s}$ - симетричність;
\item $\mdl{\mdl{\min\set{s,t}}}_{i,j=1,\ldots,n}$ - невід’ємна визначна (раніше було).
\end{itemize}
\end{exs}
\begin{exs}
\begin{equation}
X(t) = \system{0, t<\tau \\ 1, t>\tau}
\end{equation}
\begin{equation}
m(t) = \mEx{X(t)} = 0 \mP\set{X(t)=0} + 1\cdot \mP\set{x(t) = 1} = \cfrac t{90}
\end{equation}
Тепер шукаємо кореляційну функцію:
\begin{equation}
C(s,t) = \mEt{X(t)\cdot X(s)} - \mEt{X(t)} \mEt{X(s)} = \cfrac{\min\set{s,t}}{90} - \cfrac{st}{8100}
\end{equation}
\end{exs}
\section{Гаусовський процес} \marginpar{\framebox{16.04.2014}}
Процес $X(t)$ називається \textbf{гаусовським}, якщо:
\begin{equation}
\forall n\in\mN;t_1,\ldots,t_n\in T: \vect{X(t_1)\\\vdots\\ X(t_n)} - \text{гаусовский}
\end{equation}
\subsection{Критерій гаусовості випадкового вектора}
\begin{teor}
Вектор $\vxi = \vect{\xi_1\\\vdots\\\xi_n}$ є гаусовським тоді і тільки тоді, коли кожна лінійна комбінація вигляду $c_1\xi_1+\ldots+c_n\xi_n$ є гаусовою величиною.
\end{teor}
\begin{proof}
$\Rightarrow$\\
Якщо вектор дійсно гаусовський, то $c_1\xi_1+\ldots+c_n\xi_n$ - це лінійне перетворення, а отже, гаусовське.\\
$\Leftarrow$\\
Нехай $\forall\set{c_1,\ldots,c_n}:c_1\xi_1+\ldots+c_n\xi_n$ - є гаусівською.\\
\begin{equation}
\mEt{c_1\xi_1+\ldots+c_n\xi_n} = c_1a_1+\ldots+c_na_n = (\vec c,\vec a)
\end{equation}
$B$ - кореляційна матриця $\vxi$.
\begin{multline}
\mDt{c_1\xi_1+\ldots+c_n\xi_n} = \mDt{(\vec c,\vec a)} = \text{кор. матриця вектора } (\vec c,\vec a) = {} \\ {} = \overleftarrow{c}B_{\vxi} \vec c = (B_{\vxi} \vec c,\vec c)
\end{multline}
Запишемо тепер характеристичну функцію нашої комбінації. \\
Згадаємо формулу:
\begin{equation}
\gam\sim\aleph{a.\sigma^2}: \chi_{\gamma} (t)= e^{-iat - \frac{\sigma^2 t^2}{2}}
\end{equation}
А тепер запишемо формулу у нашому випадку:
\begin{equation}
\chi_{c_1\xi_1+\ldots+c_n\xi_n} (t) = e^{i(\vec c,\vec a) t - \frac{\cb{B_{\vxi} \vec c,\vec c}^2t^2}{2}}
\end{equation}
\begin{equation}
\chi_{c_1\xi_1+\ldots+c_n\xi_n} (1) = e^{i(\vec c,\vec a) - \frac{\cb{B_{\vxi} \vec c,\vec c}^2}{2}}
\end{equation}
З іншого боку:
\begin{equation}
\chi_{c_1\xi_1+\ldots+c_n\xi_n} (1) = \mEt{e^{1\cdot i \cdot \cb{c_1\xi_1+\ldots+c_n\xi_n}}} = \mEt{e^{i\cb{\vec c,\vxi}}} = \chi_{\vxi} (\vec c)
\end{equation}
Отже, $\vxi$ - гаусовський вектор.
\end{proof}
\begin{nasl}
Процес $\cb{X(t),t\in T}$ є гаусовським тоді і тільки тоді, коли:
\begin{equation}
\forall n\in\mN:,t_1,\ldots,t_n\in T,c_1,\ldots,c_n \in \mR: \suml_{i=1}^n c_i X(t_i) - \text{гаусовський}
\end{equation}
\end{nasl}
Гаусовський процес повністю визначається за $m(t)$ та $C_x(s,t)$.
\section{Вінеровський процес (Броунівський рух)}
$S_n$ - положення частки у рідині у момент дискретного часу $n$.\\
$\set{\xi_i,i\geq 1}$ - незалежні та мають такий розподіл:
\begin{center}
\begin{tabular}{|c|c|c|}
\hline
$\xi$ & -1 & 1 \\
\hline
$\mP$ & $\dfrac12$ & $\dfrac12$\\
\hline
\end{tabular}
\end{center}
Процес $(S_n,n\geq 0)$\\
\begin{equation}
m(n) = \mEt{S_n} = \mEt{\xi_1+\ldots+\xi_n} = 0
\end{equation}
\begin{multline}
C_s(n_1,n_2) = \mdl{n_1\leq n_2} = \mEt{\cb{S_{n_1} - m(n_1)}\cb{S_{n_2}-m(n_2)}} ={}\\{}= \mEt{S_{n_1}\cdot S_{n_2}} = \mEt{\cb{\xi_1+\ldots+\xi_{n_1}}\cb{\xi_1+\ldots+\xi_{n_2}}} = {} \\ {} =\mEt{x_1+\ldots+\xi_{n-1}}^2 + \mEt{\cb{\xi_1+\ldots+\xi_{n_1}}\cb{\xi_{n_1+1}+\ldots+\xi_{n_2}}} ={} \\ {}=\mDt{\xi_1+\ldots+\xi_{n_1}} = n_1
\end{multline}
В загальному випадку:
\begin{equation}
C_s(n_1,n_2) = \min\set{n_1,n_2}
\end{equation}
%графік траекторії випадкового блукання
Новий процес $X(t) = \cfrac{1}{c}S\cb{tc^2}$\\
\begin{equation}
\mEt{X(t)} = \cfrac1c \mEt{S\cb{ t{c^2}}} = 0
\end{equation}
\begin{equation}
C_x(s,t) = \mEt{\cfrac1c S\cb{s{c^2}}\cfrac1c S\cb{t{c^2}}} = \cfrac1{c^2} \mEt{S\cb{ s{c^2}} S\cb{ t{c^2}}} = \cfrac1{c^2} \min\set{sc^2,tc^2} = \min\set{s,t}
\end{equation}
При одночасному масштабуванні часу в $c^2$ разів в простору в $c$ разів, в нового процесу $X(t)$ зберігаються такі самі математичні сподівання і кореляційна функція як і у не масштабованого процесу.\\
Спрямуємо $c$ до нескінченності:
\begin{equation}
X(t)=?-\liml_{c\to+\infty} \cfrac1c S\cb{c^2t}
\end{equation}
\textbf{Вінерівським процесом} або \textbf{броунівським рухом} називають процес $\cb{W(t),t>0}$, який задовольняє три властивості:
\begin{enumerate}
\item $W(t)$ - гаусовський процес;
\item $\mEt{W(t)}=0$;
\item $C_W(s,t) = \mEt{W(s),W(t)} = \min\set{s,t}$;
\end{enumerate}
\subsection{Властивості Вінерівського процесу}
\begin{enumerate}
\item Вінерівський процес існує.\\
	$\min\set{s,t}$ - невід’ємно визначений та симетричний.\\
	Отже, можна побудувати в $\mR^n$ відповідний розподіл з математичним сподіванням $\vec 0$ та з матрицею $\mdl{\mdl{\min{t_1,t_2}}}$. Отже, 			за теоремою Колмагорова можна побудувати такий випадковий процес.
\item $w(0)=0$ майже напевно. Тобто $\mP\set{w(0)=0}=1$\\
	\begin{proof}
	$\mEt{w(0)}0$\\
	$\mEt{w^2(0)} = \mEt{w(0)w(0)} = \min{0,0}=0$\\
	$\Rightarrow \mP\set{w(0)=0}=1$
	\end{proof}
%графік Вінерівскього процесу 
\item Глянемо на $W(t)$ через лупу. \\
	\begin{equation}
	\tilde{w}(t) = \cfrac1c w(c^2t)
	\end{equation}
	Стверджується, що це також вінерівський процес.\\
	\begin{enumerate}
	\item $\mEt{\tilde w(t)} = \cfrac1c \mEt{w\cb{c^2t}} = 0$
	\item $C_{\tilde w} (t) = \mEt{\tilde{w}(s),\tilde{w}(t)} = \cfrac1{c^2} \min\set{sc^2,tc^2} = \min\set{s,t}$
	\item Чому цей процес знову гаусовський?\\
		\begin{equation}
		\vect{\tilde w(t_1)\\\vdots\\\tilde w(t_n)} = \cfrac1c\vect{w(t_1c^2)\\\vdots\\w(t_nc^2)} -\text{гаусовський}
		\end{equation}
	\end{enumerate}
\item $w(t+h)-w(t)\sim\aleph\cb{0,h}$
	\begin{proof}
	Те, що ця величина гаусівськая очевидно, оскільки вона є різницею двох величин, які утворюють гаусовський вектор.\\
	\begin{equation}
	\mEt{w(t+h)-w(t)} = 0
	\end{equation}
	\begin{multline}
	\mDt{w(t+h)-w(t)} ={} \\ {} = \mEt{\cb{w(t+h)-w(t)}^2} = \mEt{w^2(t+h)}+\mEt{w^2(t)} - {} \\ {} - 2\mEt{w(t+h)w(t)} = h
	\end{multline}
	\end{proof}	
\item Вінерівський процес має незалежні прирости. \\
	$0\leq s_1\leq t_1 \leq \ldots\leq s_n\leq t_n$\\
	Тоді $w(t_1)-w(s_1),\ldots,w(t_n)-w(s_n)$ є незалежними величинами у сукупності
	\begin{proof}
	$s_i\leq t_i \leq s_j \leq t_j$\\
	\begin{multline}
	\mEt{\cb{w(t_i)-w(s_i)}\cb{w(t_j)-w(s_j)}} = {} \\ {} = \mEt{w(t_i)w(t_j)} - \mEt{w(t_i)w(s_j)} - \mEt{w(s_i)w(t_j)} +\mEt{w(s_i)w(s_j)} 			={} \\ {} = t_i-t_i - 			s_i + s_i = 0
	\end{multline}
	Отже, вони некорельовані. А в силу гаусовості вектора $\vect{w(t_1)-w(s_1)\\\vdots\\w(t_n)-w(s_n)}$, вони тоді й незалежні.
	\end{proof}
\end{enumerate}
\section{Просунуті задачі вінерівського процесу} \marginpar{\framebox{30.04.2014}}
\subsection{Траєкторії вінерівського процесу}
\subsubsection{Вінерівський процес має неперервні траєкторії майже напевно}
Тобто, стверджується, що:
\begin{equation}
\mP\set{w(t) \in C\left[0;+\infty\right)}=1
\end{equation}
Маючи систему скінченно вимірних розподілів, неможливо визначити, чи будуть траєкторії майже напевно неперервними.\\
\begin{exs}
Розглядаємо два процеси: $X(t),Y(t),t\in\bb{0,1}$\\
$X(t) \equiv 0$ - неперервна траєкторія.\\
$Y(t) = \system{0,t\neq \tau\\ 1, t= \tau},\tau\sim U\cb{\bb{0,1}}$\\
Ці розподіли випадково рівні. 
\begin{multline*}
\mP\set{Y(t_1)=0,\ldots,Y(t_n)=0} ={}\\{}= \mP\set{\tau\in\cb{t_1,\ldots,t_n}} = 0 ={}\\{}= \mP\set{X(t_1)=0,\ldots,X(t_n)=0},\forall n,\set{t_1,\ldots,t_n}\in\bb{0,1}
\end{multline*}
\end{exs}
Після довгих та впевнених доведення невідомо кому невідомо кого, отримали таке формулювання:
\begin{teor}
Існує версія вінерівського процесу з неперервними траєкторіями.
\end{teor}
Розглянемо деякий допоміжний факт:
\begin{teor}[Колмогорова про неперервність траєкторії]
$X(t)$ - випадковий процес. Просто випадковий процес.\\
І виконується така умова:
\begin{equation*}
\exists \al,\beta>0: \mEt{\mdl{x(t)-x(s)}^\al}\leq C\mdl{t-s}^{1+\beta},\forall s,t\in T,C=C(\al,\beta)
\end{equation*}
Тоді існує версія процесу $X(t)$ з майже напевно неперервними траєкторіями.
\end{teor}
\begin{proof}
Доведення не буде, бо воно дуже складне. 
\end{proof}
\begin{proof}[Доведення основної теореми]
Перевіримо умову, яка необхідна для виконання $\al=2$
\begin{equation}
\mEt{\cb{w(t)-w(s)}^2} = \mEt{\aleph^2\cb{0,t-s}} = |t-s|^1 ?1\neq 1 +\beta,\beta>0
\end{equation}
Не пішло. Візьмемо $\al=4$:
\begin{equation}
\mEt{\cb{w(t)-w(s)}^4} = \mEt{\aleph^4\cb{0,t-s}} = |t-s|^2\mEt{\aleph^4\cb{0,1}} = 3|t-s|^2
\end{equation}
Отже, умова теореми Колмогорова виконалася з $\al=4,\beta=1$.
\end{proof}
\subsubsection{Варіація та довжина вінерівського процесу}
\begin{equation}
\Varl_{x\in\bb{a,b}} f(x) = \sup\limits_{\Delta} \suml_{k=1}^n \mdl{f(x_k) - f(x_{k+1})}
\end{equation}
\begin{exs}
$\Var\limits_{x\in\bb{a,b}} \sin x = 1+2+1 = 4$
\end{exs}
\begin{teor}
\begin{equation}
\mP\set{\Varl_{x\in\bb{a,b}} w(t)=\infty} = 1
\end{equation}
\end{teor}
Квадратична варіація:
\begin{equation}
\Vartl_{x\in\bb{a,b}} f(x) = \liml_{\delta\to0} \sup\limits_{\Delta,\max\mdl{x_k-x_{k-1}}\leq\delta} \suml_{k=1}^n \mdl{f(x_k)-f(x_{k-1})}^2
\end{equation}
\begin{teor}
Якщо функція $f(x)$ має $\Varl_{x\in\bb{a,b}} <\infty, f\in C\cb{\bb{a,b}}$, то $\Vartl_{x\in\bb{a,b}} f(x) = 0$
\end{teor}
\begin{proof}
Функція $f(x)$ рівномірно неперервна. Отже
\begin{equation}
\forall \eps,\exists \delta=\delta(\eps): \mdl{x-y}\leq \delta \Rightarrow \mdl{f(x)-f(y)} \leq \eps
\end{equation}
Зафіксуємо $\eps$ і беремо такі розбиття, що $\max\mdl{x_k-x_{k-1}} < \delta$\\
\begin{multline}
\suml_{k=1}^n \mdl{f(x_k) - f(x_{k-1})}^2 \leq \eps \suml_{k=1}^n \mdl{f(x_k)-f(x_{k-1})} \leq \eps \Varl_{x\in\bb{a,b}} f(x) 
\end{multline}
Отже, отримали:
\begin{equation}
\liml_{\delta\to0} \suml_{k=1}^n \mdl{f(x_k) - f(x_{k-1})}^2 \leq \eps \Varl_{x\in\bb{a,b}} f(x) \xrightarrow[\eps\to 0]{} 0
\end{equation}
\end{proof}
Але для вінерівського процесу $\Vartl w(t)\neq 0$.
\begin{teor}[Теорема Леві про квадратичну варіацію вінерівського процесу]
Якщо в нас є послідовність розбиттів відрізку \bb{a,b} такий, що $\max\limits_{k=0,\ldots,n} \mdl{x_k-x_{k-1}} \to 0$, то
\begin{equation}
\mL_2-\suml_{k=1}^n \cb{w(x_k) - w(x_{k-1})}^2 \to b-a
\end{equation}
\end{teor}
\begin{proof}
Перевіримо умову критерію $\mL_2$ збіжності до константи\\
\begin{eqnarray}
&\mEt{\xi_a} \to\const\\
&\mDt{\xi_a} \to 0
\end{eqnarray}
\begin{equation}
\mEt{\suml_{k=1}^n \cb{w(x_k) - w(x_{k-1})}^2} = \suml_{k=1}^n \cb{x_k - x_{k-1}} = b-a
\end{equation}
\begin{multline}
\mDt{\suml_{k=1}^n \cb{w(x_k) - w(x_{k-1})}^2} = \suml_{k=1}^n \mDt{\cb{w(x_k) - w(x_{k-1})}^2} ={}\\{}= \suml_{k=1}^n \mEt{\cb{w(x_k) - w(x_{k-1})}^4} - \suml_{k=1}^n \mEt{\cb{w(x_k) - w(x_{k-1})}^2}^2 ={}\\{}= 3\suml_{k=1}^n \cb{x_k-x_{k-1}}^2 -\suml_{k=1}^n \cb{x_k-x_{k-1}^2} = 2\suml_{k=1}^n \cb{x_k-x_{k-1}}^2 \to0
\end{multline}
Отже, з цим умов випливає, що:
\begin{equation}
\mL_2-\suml_{k=1}^n \cb{w(x_k) - w(x_{k-1})}^2 \to b-a
\end{equation}
\end{proof}
Отже, ми довели такий факт: $\mP\set{\Vartl_{\bb{a,b}} w = 0} = 0$\\
Отже, з урахуванням попередньої теореми 
\begin{equation}
\mP\set{\Var_{\bb{a,b}} w <\infty} = 0
\end{equation}
або 
\begin{equation}
\mP\set{\Var_{\bb{a,b}} w = \infty} = 1
\end{equation}
\paragraph{Висновок:} Інтеграли вигляду $\intl_a^b f(t)\dif w(t)$ не можна визначити потраєкторно і потрібний більш складний підхід.
\subsubsection{Ліричний відступ}
Якщо $\mdl{f'}\leq C$, то
\begin{equation}
\Var\leq C\sup \suml_{k=1}^n \cb{x_k-x_{k-1}} = C\cb{b-a} \leq \infty
\end{equation}
Отже, $\mP\set{w\not\in C^1 \cb{\bb{a,b}}}=1$
\begin{teor}
Для будь-якої точки $\forall t\geq 0$.
\begin{equation}
\mP\set{\exists w'(t)}  =0
\end{equation}
\end{teor}
\begin{proof}
Якщо б існувала похідна 
\begin{equation}
w'(t) = \liml_{h\to0} \cfrac{w(t+h)-w(t)}{h}
\end{equation}
\begin{equation}
\mDt{\cfrac{w(t+h)-w(t)}{h}} = \cfrac{h}{h^2} = \cfrac1h
\end{equation}
\begin{equation}
\cfrac{w(t+h)-w(t)}{h} \sim\aleph\cb{0,\cfrac1h}
\end{equation}
\begin{equation}
\chi_{\frac{w(t+h)-w(t)}{h}} (x) = e^{-\frac{x^2}h} \xrightarrow[h\to0]{} 0
\end{equation}
Але 0 не характеристична функція, оскільки $\chi(0)=1$
\end{proof}
\begin{teor}
\begin{equation}
\mP\set{\forall t\geq 0: \not\exists w'(t)} = 1
\end{equation}
\end{teor}
\section{Пуасонівський потік та процес} \marginpar{\framebox{14.05.2014}}
\subsection{Потік Пуассона}
%графік
\begin{equation}
	N(s,t) = \# \set{i:t^\ast_i \in \bb{s,t}}
\end{equation}
\begin{enumerate}
	\item Стаціонарність: 
	\begin{equation}
		\forall s,t,h\geq 0: N(s,t) = N(s+h,t+h)
	\end{equation}
	\item Відсутність післядії:
	\begin{equation}
		s_1\leq t_1 \leq\ldots\leq s_n\leq t_n: N(s_1,t_1),\ldots,N(s_n,t_n) - \text{Незалежні в сукупності}
	\end{equation}
	\item Ординарність:
	\begin{equation}
		\mP\set{N(o,h)\geq 2} = o(h),h\to 0
	\end{equation}
\end{enumerate}
Якщо виконані ці три умови, то цей потік називається {\bf потік Пуассона}
%приклади, вставити з лекцій минулого року
\begin{teor}
	$\forall s<t, N(s,t) \sim Pois(\la\cdot (t-s)),\la>0$\\
	$\la$ - деякий параметр, інтенсивність потоку
\end{teor}
\begin{teor}
	$\exists \la>0: \mP\set{N(0,h)=1}=\la\cdot h + o(h),h\to 0$
\end{teor}
\begin{proof}
	\begin{eqnarray}
		&\mP\set{N(0,h)=0}=P_0(h)
	\end{eqnarray}
	\begin{multline}
		P_0(1) = \Theta = \mP\set{N(0,1)=0}  ={}\\{}= \mP\set{\bigcap\limits_{k=1}^n N\cb{\cfrac{k-1}{n},\cfrac{k}{n}} = 0} = \prod\limits_{k=1}^n \mP\set{N\cb{\cfrac{k-1}{n},\cfrac{k}{n}}= 0} ={}\\{}= \prod\limits_{k=1}^n \mP\set{N\cb{o,\cfrac1n}=0} = P_0^n\cb{\cfrac1n} 
	\end{multline}
	\begin{equation}
		P_0\cb{\cfrac kn} = \prod\limits_{t=1}^k P_0\cb{\cfrac kn} = \Theta^{k/n}		
	\end{equation}
	Нехай тепер є $t$ - довільне дійсне додатне.
	\begin{equation}
		r_1^- \leq \ldots\leq t \leq \ldots \leq r_1^+
	\end{equation}
	Отримали дві послідовності: $\set{r_n^-,n\in \mN},\set{r_n^+,m\in \mN},r_i^-,r_i^+\in\mQ$\\
	$P_0(t)$ - ймовірність того, що не відбудеться жодної події на проміжку від нуля до t.\\
	\begin{equation}
		P_0(r_n^-) \geq P_0(t) \geq P_0(r_n^+)
	\end{equation}
	За правилом трьох міліціонерів отримуємо, що 
	\begin{equation}
		\liml_{n\to\infty} P_o(t) = \Theta^t
	\end{equation}
	Звідси очевидним образом можна отримати, що:
	\begin{equation}
		P_o(t) = \Theta^t
	\end{equation}
	$0<\Theta<1: \Theta = e^{-\la}$\\
	Отримали тоді таку формулу:
	\begin{equation}
		P_0(t) = e^{-\la t}
	\end{equation}
	Тоді 
	\begin{multline}
		P_1(h) = {} \\ {} = \mP\set{N(0,h)=1} = 1 - \mP\set{N(0,h)=0} - \mP\set{N(0,h)\geq 2} = 1 - e^{-\la h} - o(h) = {} \\ {} = \mdl{\cfrac{1-e^{-\la h}}{\la h} \xrightarrow[h\to0]{} 1 } = \la \cdot h + o(h) - o(h)
	\end{multline}
\end{proof}
\begin{proof}[Доведення попередньої теореми]
	\begin{equation}
		P_k(t) = \mP\set{N(0,h)=k},k\geq 0
	\end{equation}
	\begin{multline}
		P_{k+1}(t+h) = {} \\ {} =  \mP\set{N(0,t+h)=k+1} = \suml_{l=0}^{k+1} \mP\set{N(0,t)=l}\cdot \mP\set{N(t,t+h) = k+1-l} = {} \\ {} = P_{k+1}(t)\cdot P_0(h) + P_k(t)P_1(h) + o(h) = P_{k+1}(t)\cdot\cb{1-\la h + o(h)} + P_k(t)\cb{\la h +o(h)} + o(h) ={} \\ {} = -\la h P_{k+1}(t) + \la h P_k(t) + o(h)
	\end{multline}
	Отже, отримали:
	\begin{equation}
		\cfrac{P_{k+1}(t+h)-P_{k+1}(t)}{h} = -\la  P_{k+1}(t) + \la P_k(t) + o(1)
	\end{equation}
	\begin{equation}
		h\to0:P_{k+1}' = -\la P_{k+1} + \la P_k
	\end{equation}
	Складемо систему рівнянь:
	\begin{eqnarray}
		&P_0(t) = e^{-\la t}\\
		&P_{k+1}' = -\la P_{k+1} + \la P_k\\
		&P_0(k) = \system{0,k\neq 0\\1,k=0}
	\end{eqnarray}
	Зробимо заміну функції: $Q_k(t) = e^{\la t} P_k(t)$\\
	А зворотною буде заміна: $P_k(t) = e^{-\la t} Q_k(t)$
	\begin{equation}
		P_{k+1}'(t) = -\la e^{-\la t} Q_{k+1}(t) + e^{-\la t} Q_{k+1}'(t)
	\end{equation}
	А тепер підставимо це добро:
	\begin{equation}
		-\la e^{-\la t} Q_{k+1} + e^{-\la t} Q_{k+1}' = -\la e^{-\la t} Q_{k+1} + \la e^{- \la t} Q_k
	\end{equation}
	Залишилося після скорочення таке:
	\begin{equation}
		Q_{k+1}'(t) = \la Q_k(t)
	\end{equation}
	З такими початковими умовами:
	\begin{eqnarray}
		&Q_0(t) = 1\\
		&Q_k(0) = \system{0,k\neq 0\\1,k=0}
	\end{eqnarray}
	Починаємо розв’язувати знизу:
	\begin{eqnarray}
		&Q_0(t)=1\\
		&\Rightarrow Q_1'(t) = \la \Rightarrow Q_1(t) = \la t \\
		& \vdots\nonumber\\
		&Q_k(t)  = \cfrac{\la^k t^k }{k!}
	\end{eqnarray}
	Тоді
	\begin{equation}
		P_k(t) = e^{-\la t}\cfrac{\la^k t^k }{k!}
	\end{equation}
	Тоді можна легко отримати:
	\begin{equation}
		\mP\set{N(s,t) = k } = e^{-\la (t-s)} \cfrac{\la^k\cb{t-s}^k}{k!}
	\end{equation}
\end{proof}
\subsection{Пуассонівський процес}
%милий графік
Процес Пуассона визначаємо за такою формулою:
\begin{equation}
	N(t) = N\cb{\bb{0,t}}
\end{equation}
Потік Пуассона має мати такі властивості:
\begin{itemize}
	\item $\mP\set{N(0) = 0}=1$ 
	\item $N(t) - N(s) \sim Pois\cb{\la\cb{t-s}}$
	\item прирости $ N(t_1)-N(s_1),\ldots,N(t_n)-N(s_n)$ - незалежні в сукупності, якщо $s_1\leq t_1\ldots\leq s_n\leq t_n$
	\item $N(t)$ - неперервний з правої сторони та має границі з лівої сторони.
\end{itemize}
\paragraph{Математичне сподівання}
\begin{equation}
	\mEt{N(t)} = \mEt{N(t) - N(0)} = \mEt{Pois(\la t)} = \la t 
\end{equation}
\paragraph{Кореляційна функція}
\begin{multline}
	\mEt{\cb{N(s)-\la s}\cb{N(t) - \la t}} ={}\\{}= |s\leq t| = \mEt{\cb{N(s)-\la s}\cb{N(s) - \la s +N(t) - N(s) - \la \cb{t-s}}} ={}\\{}=  \mEt{\cb{N(s)-\la s}\cb{N(s)-\la s}} + \mEt{\cb{N(s)-\la s}\cb{N(t)-N(s)-\la\cb{t-s}}} ={}\\{}= \mDt{Pois(\la s)} + 0 = \la s = \la \min\set{s,t}
\end{multline}

\subsection{Час між подіями}\marginpar{\framebox{21.05.2014}}
Введемо нові величини, які будуть позначати час подій $\set{\tau_n,n\in\mN}$ - \textit{inter-arrival times}.
\begin{teor}
	$\set{\tau_n,n\in\mN}$ - незалежні та розподілені за $Exp(\la)$.
\end{teor}
\begin{proof}
	Доведемо лише частину з цієї великої теореми.
\end{proof}
\begin{teor}[Еквівалентне формулювання]
	Нехай $\set{\tau_i,i\in\mN}$ - послідовність незалежних експоненціально $\cb{Exp\cb{\la}}$ розподілених випадкових величин. 	За ними побудуємо процес стрибків у заданий час.\\
	Тоді такий процес буде Пуассонівським.
\end{teor}
\begin{proof}
	Назвемо побудований процес $N(t)$. Тоді 
	\begin{equation}
		N(t) = \max \set{n\geq0: T_n\leq t},\quad T_n = \suml_{i=1}^n t_i,T_0=0
	\end{equation}
	Подивимося, які властивості явно виконуються:
	\begin{enumerate}
		\item $N(0)=0$
		\item Неперервність справа і має стрибки зліва.
	\end{enumerate}
	Головне довести інші властивості.\\
	Розглянемо $0\leq s_1\leq t_1\leq\ldots\leq s_k\leq t_k$ і прирости: $N(t_1)-N(s_1),\ldots,N(t_k)=N(s_k)$. Необхідно довести їх незалежність та правильний розподіл. Отже, потрібно довести, що:
	\begin{equation}
		\mP\set{\vect{ N(t_1)-N(s_1) \\\vdots\\ N(t_k)=N(s_k) } = \vect{\Delta_1\\\vdots\\\Delta_k}} = \cfrac{e^{-\la\cb{t_1-s_1}} \cb{\la\cb{t_1-s_1}}^{\Delta_1}}{\Delta_1!} \cdot \cfrac{e^{-\la\cb{t_k-s_k}} \cb{\la\cb{t_k-s_k}}^{\Delta_k}}{\Delta_k!} 
	\end{equation}
	Ми хочемо це довести, але не будемо. А доведемо для $k=1$, а ще для простоти покладемо $s_1=0$.\\
	Отже, ми хочемо довести таку річ:
	\begin{equation}
		\mP\set{N(t)=\Delta} = e^{-\la t} \cfrac{\cb{\la t}^\Delta}{\Delta!}
	\end{equation}
	\begin{multline}
		\mP\set{N(t)=\Delta} = \mP\set{N(t)\geq\Delta} - \mP\set{N(t)\geq\Delta+1} = \mP\set{T_\Delta\leq t} -{}\\{}- \mP\set{T_{\Delta+1}\leq t} = \mP\set{\tau_1 + \ldots + \tau_\Delta\leq t} - \mP\set{\tau_1 + \ldots + \tau_{\Delta+1}\leq t}
	\end{multline}
	$\tau_1,\ldots,\tau_\Delta\sim Exp(\la)$ тоді\\
	$\tau_1+\ldots+\tau_\Delta\sim G(\Delta,\la)$
	Отже, отримуємо таку щільність розподілу:
	\begin{equation}
		f_{\tau_1+\ldots+\tau_\Delta} (x) = \system{ \cfrac{\la^\Delta}{\cb{\Delta-1}!} x^{\Delta-1} e^{-\la x},x>0 \\ 0,x\leq 0}
	\end{equation}
	Тоді:
	\begin{equation}
		\mP\set{\tau_1+\ldots+\tau_\Delta\leq t} = \intl_0^t \cfrac{\la^\Delta}{\cb{\Delta-1}!} x^{\Delta-1} e^{-\la x} \dx = I_1
	\end{equation}
	Для другого доданка те саме:
	\begin{equation}
		\mP\set{\tau_1+\ldots+\tau_\Delta\leq t} = \intl_0^t \cfrac{\la^{\Delta+1}}{\cb{\Delta}!} x^{\Delta} e^{-\la x} \dx = I_2
	\end{equation}
	Віднімемо їх:
	\begin{equation}
		\mP\set{\tau_1 + \ldots + \tau_\Delta\leq t} - \mP\set{\tau_1 + \ldots + \tau_{\Delta+1}\leq t} = I_1 - I_2
	\end{equation}
	Давайте трошки розважимося з цими інтегралами:
	\begin{multline}
		I_1 - I_2 = \intl_0^t \cfrac{\la^\Delta}{\cb{\Delta-1}!} x^{\Delta-1} e^{-\la x} \dx -  \intl_0^t \cfrac{\la^{\Delta+1}}{\cb{\Delta}!} x^{\Delta} e^{-\la x} \dx ={}\\{}=  \cfrac{\la^\Delta}{\cb{\Delta-1}!} \intl_0^t x^{\Delta-1} e^{-\la x} \dx - \cfrac{\la^{\Delta+1}}{\cb{\Delta}!} \cb{ \left. -\cfrac{x^\Delta}{\la} e^{-\la x} \right|_0^t + \cfrac1\la \intl_0^t e^{-\la x} \Delta x^{\Delta-1} \dx}	 = {} \\ {} = \cfrac{\la^\Delta}{\cb{\Delta-1}!} \intl_0^t x^{\Delta-1} e^{-\la x} \dx +\cfrac{\la^\Delta}{\Delta!} t^\Delta e^{-\la t} - \cfrac{\la^\Delta}{\cb{\Delta-1}!} \intl_0^t e^{-\la x} x^{\Delta-1} \dx = {} \\ {} =  \cfrac{\la^\Delta}{\Delta!} t^\Delta e^{-\la t} = \mP\set{Pois(\la t) = \Delta}
	\end{multline}
\end{proof}
\section{Елементи актуарної математики}
\textbf{Актуарна математика} - це математика страхових компаній. \\
Під цими елементами буде основна та найбільш проста модель - модель Крамера-Лундберга, яка виникла у 20-30-тих роках 20-того сторіччя.\\
Логічно припустити, що страхові випадки утворюють потік Пуассона з деякою інтенсивністю $\la$. І в нас виникає такий собі \textbf{процес ризику} - процес, який описує поточний капітал страхової компанії.
\begin{equation}
	U(t) = u + ct - \suml_{i=1}^{N(t)} X_i
\end{equation}
с - компанія отримує в одиницю часу. $X_i$- claims, страхові виплати - деякі незалежні, однаково розподілені випадкові величини такі, що $X_i \geq 0$\\
Банкрутство(In) - це $\set{\exists t\geq 0: U(t)<0}$ на нескінченному часовому горизонті. \\
Введемо ймовірність банкрутства:
\begin{equation}
	\psi(u)= \mP\set{\exists t\geq 0: U(t)<0}
\end{equation}
Знайдемо $\psi(u)$, $u$ - початковий капітал.
\begin{multline}
	\mEt{U(t)} = u + ct - \mEt{\suml_{i=1}^{N(t)} X_i} = {}\\{}= u + ct - \mEt{\mEt{\suml_{i=1}^{N(t)}  X_i\diagup_{N(t)=k}}}  = u + ct - \mEt{N(t)\cdot \mEt{X}}  ={}\\{}= u + ct - \mEt{N(t)}\cdot \mEt{X} = u + ct - \la m t, \quad m = \mEt X 
\end{multline}
Перепишемо цю формулу у такому вигляді:
\begin{equation}
	\mEt{U(t)} = u + \cb{c-\la m}t,\quad t\geq 0
\end{equation}
Якщо $c\leq\la m$, то $\forall u:\quad\psi(u)=1$\\
Якщо $c>\la m$, то $\forall u:\quad\psi(u)<1$\\
Умова $c>\la m$ називається \textbf{умова чистого прибутку} або в англомовній літературі "\textbf{NPC(Net Profit Condition)}".\\
Умову банкрутства можна переписати так:
\begin{multline}
In = \set{\inf\limits_{t\geq 0} U(t)<0} = \set{u + \inf\limits_{t\geq 0} \cb{ct - \suml_{i=1}^{N(t)} X_i} < 0 } ={} \\ {} =  \set{u + \inf\limits_{n\in \mN} \cb{c T_n + \suml_{i=1}^n X_i } < 0 }  = {} \\ {}= \set{u +\inf\limits_{n\in\mN} \suml_{i=1}^n \cb{c\tau_i - X_i}<0} = \set{\inf\limits_{n\in\mN} \suml_{i=1}^n \cb{c\tau_i - X_i} < - u}
\end{multline}
$\set{\tau_i,i\in\mN}$ - незалежні $Exp(\la)$\\
Згідно до ЗВЧ 
\begin{equation}
	\cfrac{\suml_{i=1}^n \cb{c\tau_i-X_i}}{n} \xrightarrow[n\to\infty]{}\mEt{c\tau_i - X_i} = \cfrac c\la - m
\end{equation}
Отже, якщо $\cfrac c\la - m<0$, то $\cfrac{\suml_{i=1}^n \cb{c\tau_i-X_i}}{n}$ прямує до деякого від’ємного числа і тоді $\suml_{i=1}^n \cb{c\tau_i-X_i}\to-\infty$.\\
Для рівності довести складно, але повірте - це щира правда!\\
Якщо $c<\la m$, то $\cfrac{\suml_{i=1}^n \cb{c\tau_i-X_i}}{n}$ буде прямувати до якогось додатного числа і $\suml_{i=1}^n \cb{c\tau_i-X_i}\to+\infty$
\section{Інтегральне рівняння для ймовірності небанкрутства}
\begin{equation}
	\phi (u) = 1 - \psi (u)
\end{equation}
\begin{teor}
Якщо виконано умову NPC, то ймовірність небанкрутства $\phi(u)$ задовольняє рівнянню:
\begin{equation}
	\phi(u) = \phi(0) + \cfrac\la c \intl_0^u \phi(u-t) \overline{F}_x(t)\dt
\end{equation}
де $\overline{F}_x (t)= 1 - F_x(t) = \mP\set{x_i\geq t}$
\end{teor}
\begin{proof}
\begin{multline}
	\phi(x) = \mP\set{x_1 - c\tau_1\leq u, \sup\limits_{n\in \mN} \suml_{i=2}^n \cb{x_i-c\tau_i}\leq u +c\tau_1 - x_1}  = {}\\{}= \mEt{\mEt{\mathbf{I}\set{x_1 - c\tau_1\leq u, \sup\limits_{n\in \mN} \suml_{i=2}^n \cb{x_i-c\tau_i}\leq u +c\tau_1 - x_1}\diagup_{\cb{\tau_1,x_1}}}} = {} \\ {} = \mEt{   \mEt{\mathbf{I}\set{x_1-c\tau_1\leq u}\cdot\mathbf{I}\set{\sup\limits_{n\in \mN} \suml_{i=2}^n \cb{x_i-c\tau_i}\leq u +c\tau_1 - x_1}\diagup_{\cb{\tau_1,x_1}}}  } = {} \\ {} = \mEt{\mathbf{I}\set{x_1-c\tau_1\leq u}\cdot \mEt{\mathbf{I}\set{\sup\limits_{n\in \mN} \suml_{i=2}^n \cb{x_i-c\tau_i}\leq u +c\tau_1 - x_1}\diagup_{\cb{\tau_1,x_1}}}} = {} \\ {} = \mEt{\mathbf{I}\set{x_1-c\tau_1\leq u}\cdot \mP\set{\sup\limits_{n\in \mN} \suml_{i=2}^n \cb{x_i-c\tau_i}\leq u +c\tau_1 - x_1\diagup_{\cb{\tau_1,x_1}}}} = {} \\ {} = \mEt{\mathbf{I}\set{x_1-c\tau_1\leq u}\cdot \phi\cb{u+c\tau_1-x_1}} = {} \\ {} = \intl_0^\infty \intl_0^\infty f_x(x) \la e^{-\la t} \mathbf{I}\set{x_1-c\tau_1\leq u} \phi\cb{u+ct-x}\dx\dt = {}\\ {} =\intl_0^\infty \dt \intl_0^{u+ct} f_x(x) \la e^{-\la t} \mathbf{I}\set{x_1-c\tau_1\leq u} \phi\cb{u+ct-x}\dx
\end{multline}
Отже, ми отримали таке інтегральне рівняння:
\begin{equation}
	\phi(u) =  \la \intl_0^\infty \dt \intl_0^{u+ct} f_x(x)e^{-\la t} \mathbf{I}\set{x_1-c\tau_1\leq u} \phi\cb{u+ct-x}\dx
\end{equation}
Робимо заміну $u+ct=s:t\to s;t=\cfrac{s-u}c$.
\begin{multline}
	\phi(u) = \cfrac\la c \intl_u^\infty \dif s \intl_0^s f(x) e^{-\la\cfrac sc} e^{\la \cfrac uc} \phi(s-x)\dx = {} \\{} = \cfrac\la c e^{\la \frac uc} \intl_{s=u}^\infty \intl_{x=0}^\infty f(x) e^{-\la \frac sc} \phi(s-x) \dx\dif s
\end{multline}
\begin{multline}
	\phi'(u) = \cfrac{\la^2}{c^2} e^{\la\frac uc} \intl_{s=u}^\infty \intl_{x=0}^\infty f(x) e^{-\la \frac sc} \phi(s-x) \dx\dif s -{}\\{} - \cfrac\la c e^{\la\frac uc} \intl_0^u f(x) e^{-\la \frac uc} \phi\cb{u-x} \dx = \cfrac\la c \phi(u) - \cfrac\la c \intl_0^u f(x) \phi(u-x) \dx
\end{multline}
Ця штука має назву \textbf{інтегро-диференціальне рівняння для небанкрутства}. Загалом, нічого ідейного далі немає, а просто штучні перетворення. Залишається лише повірити.
\end{proof}
Отримали ще два дивних питання:\marginpar{\framebox{24.05.2014}}
\begin{enumerate}
\item $\phi(0)=?$
\item $x_i = Exp(\la)$ - єдина ситуація, в яких це рівняння можна чесно та аналітично розв’язати.
\end{enumerate}
Перше питання:\\
Що відбувається при $u\to+\infty:$
\begin{equation}
	1 = \phi(0) +\cfrac\la c \intl_0^\infty 1 \cdot \overline{F}(t)\dt
\end{equation}
Також можна зобразити у такому вигляді:
\begin{equation}
	1 = \phi(0) + \cfrac\la c m
\end{equation}
А отже,
\begin{equation}
	\phi(0) = 1 - \cfrac\la c m
\end{equation}
Друге питання:
\begin{equation}
	\phi(u) = 1 -\cfrac\la c m + \cfrac\la c \intl_0^u \phi(u-t) \overline{F}(t)\dt 
\end{equation}
Якщо $x_i=Exp(\al)$, то $m=\cfrac1\la, \overline{F}(t) = e^{-\al t},t\geq 0$
\begin{equation}
	\phi(u) = 1 - \cfrac{\la}{c\al} + \cfrac\la c \intl_0^u \phi(u-t) e^{-\al t} \dt
\end{equation}
Використаємо перетворення Лапласа
\begin{equation}
	\Phi(p) = \cfrac{1-\frac{\la}{c\al}}p +\cfrac\la c \Phi(p) \cfrac1{p+\al} 
\end{equation}
Перенесемо всі $\Phi(p)$ в одну частину
\begin{equation}
	\Phi(p) \cb{1 - \cfrac{\la}{c\cb{p+\al}}} = \cfrac{1-\frac{\la}{c\al}}p
\end{equation}
Ну і поділимо на коефіцієнт при $\Phi(p)$
\begin{equation}
	\Phi(p) = \cfrac{\cb{c\al -\la}  \cb{p+\al}}{c \al p\cb{\cb{p+\al} -\frac\la c}}
\end{equation}
Розкладемо на прості дроби
\begin{equation}
	\Phi(p) = \cb{1 - \cfrac{\la}{c\al}} \cb{ \cfrac{\al}{\cb{\al - \frac\la c}p} - \cfrac{\frac\la c}{\cb{\al - \frac\la c}\cb{p +\al  - \frac\la c}} }
\end{equation}
Використаємо зворотнє перетворення Лапласа
\begin{equation}
	\phi(u) = 1 - \cfrac{\la}{\al c} e^{-\cb{\al - \frac\la c}u}
\end{equation}
Тоді ймовірність банкрутства буде така:
\begin{equation}
	\psi(u) = \cfrac{\la}{\al c} e^{-\cb{\al - \frac\la c}u}
\end{equation}
\chapter{Практика}
\section{Практика}\marginpar{\framebox{05.03.2014}}
\begin{tsk}
$\xi\sim Pois\cb{\la}$ - кількість студентів. \\
Студент здає іспит з ймовірністю $p$ і не здає з ймовірністю $1-p$.\\
Потрібно довести, що кількість тих, хто склад іспит також розподілена $\sim Pois\cb{\la p}$.
\begin{multline}
\mP\set{\eta=k} = \suml_{i=k}^\infty \mP\set{\xi=i}\cdot \mP\set{\eta=k\setminus\xi=i} = \suml_{i=k}^\infty \cfrac{e^{-\la}\la^i}{i!} C^k_i p^k \cb{1-p}^{i-k} ={}\\{}= e^{-\la} p^k \suml_{i=k}^\infty \cfrac{\la^i}{i!} C^k_i \cb{1-p}^{i-k} = \begin{vmatrix}
j=i-k
\end{vmatrix} = e^{-\la} p^k \suml_{j=0}^\infty \cfrac{\la^{j+k}}{(j+k)!} C^k_{j+k} \cb{1-p}^{j} = {} \\ {}= \cfrac{e^{-\la}\la^k}{k!} p^k \suml_{j=0}^\infty \cfrac{\la^j}{j!} \cb{1-p}^j =  \cfrac{e^{-\la}\la^k}{k!} p^k \cdot e^{\la(1-p)} = \cfrac{e^{-p\la} \cb{\la p}^k}{k!}
\end{multline}
\end{tsk} 
\begin{tsk}
$\veps=\vect{\eps_1\\\eps_2\\\eps_3\\\eps_4\\\eps_5}$\\
Це \textbf{вектор білого шуму}, тобто:
\begin{itemize}
\item $\mEt{\eps_i}=0,\forall i$;
\item $\mDt{\eps_i}=1,\forall i$;
\item $\mEt{\eps_i\eps_j}=0,\forall i,j:i\neq j$.
\end{itemize}
Будуємо з нього інший вектор:
\begin{equation}
\vu = \vect{\eps_1+\eps_2+\eps_3+\eps_4+\eps_5\\2\eps_1-\eps_2-\eps+4+2\eps_5\\\eps_1+3\eps_3+\eps_5\\ \eps_2-\eps_1\\ \eps_5-\eps_4}
\end{equation}
Спостерігаються координати з непарними номерами, а знайти оцінку координат з парними номерами.
\begin{eqnarray}
\vxi=\vect{\eps_1+\eps_2+\eps_3+\eps_4+\eps_5\\ \eps_1+3\eps_3+\eps_5 \\ \eps_5-\eps_4}\\
A_{\vxi} = \begin{pmatrix}
1 & 1 & 1 & 1 & 1 \\
1 & 0 & 3 & 0 & 1\\
0 & 0 & 0 & -1 & 1
\end{pmatrix}\\
\veta = \vect{2\eps_1-\eps_2-\eps+4+2\eps_5\\ \eps_2-\eps_1}\\
B_{\vxi} = \begin{pmatrix}
2 & -1 & 0 & -1 & 2 \\
-1 & 1 & 0 & 0 & 0
\end{pmatrix}
\end{eqnarray}
Запишемо формулу оцінки:
\begin{equation}
\hveta = \vm_{\veta} + C_{\veta,\vxi} \iD_{\vxi}^{-1} \cb{\vxi-\vm_{\vxi}}
\end{equation}
З умови відомо, що:
\begin{equation}
\vm_{\veta}=\vec 0;\vm_{\vxi} = \vec 0
\end{equation}
Знайдемо дисперсійну матрицю:
\begin{multline}
\iD_{\vxi} = \mEt{\cb{\vxi-\mEvx}\mt{\cb{\vxi-\mEvx}}} = \mEt{\cb{A\veps-\mEt{A\veps}}\mt{\cb{A\veps-\mEt{A\veps}}}} = {} \\ {} =  \mEt{A\cb{\veps-\mEt{\veps}}\mt{A\cb{\veps-\mEt{\veps}}}} = \mEt{A\cb{\veps-\mEt{\veps}}\mt{\cb{\veps-\mEt{\veps}}}\mt{A}} = {} \\ {} = A\underbrace{\mEt{\cb{\veps-\mEt{\veps}} \mt{\cb{\veps-\mEt{\veps}}}}}_{\iD_{\veps}}\mt{A} = A\mt A
\end{multline}
Здогадаємося по аналогії до формули:
\begin{equation}
C_{\veta,\vxi} = B \mt A
\end{equation}
Отже, отримали:
\begin{equation}
\hveta = B\mt A \cb{A \mt A}^{-1}\vxi = \begin{pmatrix}
\frac{17}{55} & \frac1{11} & \frac{16}{11}\\
\frac2{11}& -\frac{2}{11} & \frac1{11}
\end{pmatrix}\cdot\vect{\xi_1\\ \xi_2\\ \xi_3}
\end{equation}
\end{tsk}
\begin{tsk}
Саша та Маша стріляють в лося. Кожен робить по 10 пострілів, при чому навіть у мертве тіло. Ймовірність потрапляння Саші $0.9$, а Маши $0.5$. Спостерігається загальна кількість влучень, побудувати лінійну оцінку для кількості потраплянь Саші.\\
%Хтось запишіть сюди задачу
\end{tsk}
\begin{tsk}
Випадкова величина $X \sim U(\bb{0,\cfrac\pi2})$. Спостерігається $\sin$, оцінити $\cos$.
\begin{eqnarray}
&\xi = \sin X;\eta = \cos X\\
&\mEe = \cfrac2\pi \\
&\mEx = \cfrac2\pi\\
&\mEt{\xi^2} = \cfrac12\\
&\mDx=\cfrac12 - \cfrac4{\pi^2}\\
&K_{\xi,\eta} = \cfrac1\pi - \cfrac4{\pi^2}\\
&\heta = \cfrac2\pi +\cb{\cfrac1\pi -\cfrac4{\pi^2}}
\end{eqnarray}
%Допишіть хтось
\end{tsk}
\begin{tsk}
Розподіл числа потомків є геометричним з параметром $p$.\\
Знайти ймовірність виродження гіллястого процесу.
%Перепишіть хтось
\end{tsk}
\subsection{Домашнє завдання}
\begin{tsk}\label{pr:2:1}
Маша стріляє в круглу ціль радіусу 1 метр. Точка потрапляння рівномірно розподілена в цьому колі. Спостерігається відхилення від центра по горизонталі. Знайти оптимальну лінійну оцінку відхилення по вертикалі. Відхилення обчислюється через модулі.\\
$\mdl{\xi}$ - відстань по горизонталі.\\
$\mdl{\eta}$ - відстань по вертикалі.\\
\begin{equation}
\mdl{\heta} = m_{\mdl{\eta}} + C_{\mdl{\eta},\mdl{\xi}} D^{-1}\cb{\mdl{\xi}-m_{\mdl\xi}}
\end{equation}
%очевидно, зробити самому потім
\end{tsk}
\begin{tsk}
Спостерігається обидва відхилення в задачі \eqref{pr:2:1}. Оцінити відстань від точки потрапляння до центру мішені. 
\end{tsk}
\begin{tsk}
Теж сама задача, що й \eqref{pr:2:1}, але в нас тепер немає мішені. Координати точки потрапляння це незалежні гаусовські величини. Спостерігаються відхилення від вісей, оцінити відстань від точки потрапляння до початку координат.
\end{tsk}
\begin{tsk}
Число потомків має пуасонівський розподіл. Яким має бути параметр цього розподіл, щоб ймовірність виродження дорівнювала $\cfrac12$.
\end{tsk}
\section{Практика}\marginpar{\framebox{19.03.2014}}
\begin{tsk}
\begin{tikzpicture}
\node[] (c1) {Саша};
\node[right=of c1] (c2) {Паша};
\node[right=of c2] (c3) {Даша};
\path[->]
(c1) edge node[above]{1} (c2)
(c2) edge node[above]{1}(c3)
(c3) edge[loop right] node[right]{1/2}(c3)
(c3) edge[bend left] node[below]{1/2} (c1)
;
\end{tikzpicture}
\begin{equation}
\kvp^{(0)} = \cb{1,0,0}
\end{equation}
Мінімальне $k=4$.\\
Запишемо систему:
\begin{equation}
\system{\kv{\Pi}\mP = \kv\Pi\\\Pi_1+\Pi_2+\Pi_3=1}
\end{equation}
Розпишемо цю систему:
\begin{equation}
\system{
\cfrac12\Pi_3 = \Pi_1\\ 
\Pi_1=\Pi_2\\
\Pi_2+\cfrac12\Pi_3 = \Pi_3\\
\Pi_1+\Pi_2+\Pi_3=1
}
\end{equation}
Отримали розв’язок
\begin{equation}
\kv\Pi = \cb{\cfrac14,\cfrac14,\cfrac12}
\end{equation}
Знайдемо власні числа нашої матриці $\mP$.\\
\begin{equation}
\det\cb{\mP-\la I} = \begin{vmatrix}
-\la & 1 & 0\\
0 & -\la & 1 \\
\cfrac12 & 0 & \cfrac12-\la
\end{vmatrix} = -\la^3+\cfrac{\la^2}2+ \cfrac12
\end{equation}
Розв’язавши це рівняння отримуємо:
\begin{equation}
\la = \cb{1,\cfrac14\cb{-1+i\sqrt7},\cfrac14\cb{-1-i\sqrt7}}
\end{equation}
Загалом, все сумно. Тоді спробуємо щось зробити більш адекватне. Але ні, не спробували.
\end{tsk}
\begin{tsk}
Знайте математичне сподівання момента першого повернення до Саші.\\
$\tau$ - момент першого повернення до Саші.\\
\begin{tabular}{|c|c|c|c|c|c|c|}
\hline
$\tau$ & 1 & 2 & 3  & 4 & 5 & $\ldots$\\
\hline
$\mP$ & 0 & 0 & 1/2 & 1/4 & 1/16 & $\ldots$ \\
\hline
\end{tabular}\\
\begin{equation}
\mEt\tau = \suml_{i=3}^\infty p_i = \suml_{i=3}^\infty 2^{-(k-2)} i = \suml_{l=0}^\infty 2^{-(l+1)} (l+3) = 1+3 = 4
\end{equation}
\end{tsk}
\subsection{Домашнє завдання}
\begin{tsk}
Маша підкидає монетку нескінченну кількість разів. Яка ймовірність того, що три решки підряд з’являться раніше, ніж два герби підряд.\\
Розглянемо таку схему:\\
\begin{tikzpicture}
\node[] (s1) {1.РГ};
\node[below right=of s1] (s11) {5.РГГ};
\node[below left=of s1] (s12) {2.ГР};
\node[below=of s12] (s122){3.ГРР};
\node[below=of s122] (s1222) {4.РРР};
\node[above left=of s1] (s) {Старт};
\path[->]
(s1) edge node[above]{1/2} (s11)
(s12) edge[bend left] node[above]{1/2} (s1)
(s1) edge node[below]{1/2} (s12)
(s12) edge node[left]{1/2} (s122)
(s122) edge[bend right]node[left]{1/2}(s1)
(s122) edge node[left]{1/2} (s1222)
(s) edge node[right]{1/2} (s1)
(s) edge node[left]{1/2} (s12)
;
\end{tikzpicture}\\
Отримали ймовірності: $h_1 = \cfrac15;h_2 = \cfrac25;h_3 = \cfrac35;h_4=1$\\
Тоді, за формулою повної ймовірності:
\begin{equation}
\cfrac12 \cdot h_1 + \cfrac12 \cdot h_2 = \cfrac3{10}
\end{equation}
\end{tsk}
\begin{tsk}
\begin{center}
\begin{tikzpicture}[node distance=2cm]
\node[] (c1) {2.Дом};
\node[below right=of c1] (c2) {1.Поле};
\node[above right=of c2] (c3) {3.Ручей};
\node[below=of c2] (c4) {5.Обрив};
\node[right=of c3] (c5) {4.Крокодил};
\node[above=of c1] (c6) {6.М'ясокобінат};
\path[->]
(c1) edge[bend right] node[below left]{3/5} (c2)
(c1) edge[bend left] node[above]{1/5} (c3)
(c1) edge node[left]{1/5} (c6) 
(c2) edge[bend right] node[below right]{3/5} (c3)
(c2) edge node[above right]{1/5} (c1)
(c2) edge node[left]{1/5} (c4)
(c3) edge node[above]{2/5} (c1) 
(c3) edge node[above left]{2/5} (c2)
(c3) edge node[above]{1/5} (c5)
;
\end{tikzpicture}
\end{center}
\begin{itemize}
\item Спочатку корова у домі, обчислити ймовірність кожної смерті на нескінченності.
\item Знайти математичне сподівання смерті.
\end{itemize}
\begin{equation}
h_i = \mP\set{\xi_\infty=4\setminus\xi_0=i}
\end{equation}
\begin{eqnarray}
h_4 &=& 1\\
h_5 &=& 0\\
h_6 &=& 0\\
h_1 &=& \cfrac15 h_2 + \cfrac15 \cdot 0 + \cfrac35 h_3\\
h_2 &=& \cfrac35 h_1 + \cfrac15 \cdot 0 + \cfrac15 h_3 \\
h_3 &=& \cfrac15 \cdot 1 + \cfrac25 h_1 + \cfrac25 h_2
\end{eqnarray}
Розв’яжемо цю милу систему рівнянь.\\
Отримали: $h_1 = 0.32;h_2 = 0.28;h_3 = 0.44$\\
Тепер знайдемо математичні сподівання:
\begin{eqnarray}
m_4 &=& 0\\
m_5 &=& 0\\
m_6 &=& 0\\
m_2 &=& 1 + \cfrac15\cdot 0 +\cfrac15 \cdot m_3 + \cfrac35 \cdot m_1\\
m_1 &=& 1+ \cfrac15 \cdot 0 + \cfrac35 \cdot m_3 +  \cfrac15 \cdot m_2\\
m_3 &=& 1 + \cfrac15 \cdot 0 + \cfrac25 \cdot m_2 + \cfrac25 \cdot m_1
\end{eqnarray}
Розв’яжемо це рівняння:
\begin{equation}
m_1 = m_2 = m_3 = 5
\end{equation}
\end{tsk}
\section{Практика} \marginpar{\framebox{02.04.2014}}
\begin{tsk}
\begin{tikzpicture}
\node[] (s1) {4.Саша};
\node[right=of s1] (s2) {3.Аркаша};
\node[above=of s1] (s3) {2.Маньяк 1};
\node[above=of s2] (s4) {1.Маньяк 2};
\path[->]
(s1) edge node[above]{2/3} (s2)
(s2) edge[bend left] node[below]{1/2} (s1)
(s1) edge node[left]{1/6} (s3)
(s1) edge[loop left] node[left]{1/6} (s1)
(s2) edge[loop right] node[right]{1/3} (s2)
(s2) edge node[right]{1/6} (s4);
\end{tikzpicture}\\
Питання : 
\begin{equation}
h_i = \mP\set{\xi_\infty = 1\setminus \xi_0 = i}
\end{equation}
\begin{eqnarray}
h_1 &=& 1\\
h_2 &=& 0\\
h_3 &=& \cfrac16 \cdot h_1 + \cfrac16 \cdot h_3 + \cfrac23 \cdot h_4\\
h_4 &=& \cfrac12 \cdot h_3 + \cfrac16 h_2 + \cfrac13 \cdot h_4
\end{eqnarray}
Отримали: $h_3 = \cfrac12;h_4 = \cfrac38$\\
Тепер знайдемо середній час життя:
\begin{eqnarray}
m_1 &=& 0\\
m_2 &=& 0\\
m_3 &=& 1 + \cfrac16 m_1 + \cfrac16 m_3 + \cfrac23 m_4 \\
m_4 &=& 1 + \cfrac12 m_3 + \cfrac12 m_2 + \cfrac13 m_4
\end{eqnarray}
Отримали: $m_3 = m_4 = 6$
\end{tsk}
\begin{tsk}
\begin{tikzpicture}[node distance= 3cm]
\node[circle,draw,label=below:Дом] (s1) {0};
\node[circle,draw,label=below:Нічний клуб,right=of s1] (s2) {};
\node[circle,draw,label=below:Міліція,right=of s2] (s3) {n};
\path[-]
(s1) edge (s2) edge (s3)
;
\end{tikzpicture}\\
\begin{eqnarray}
&h_0 = 1\\
&h_n = 0\\
k=1,\ldots,n & h_k = \cfrac12 h_{k+1}+ \cfrac12 h_{k-1}
\end{eqnarray}
\begin{equation}
h_k = 1 - \cfrac kn
\end{equation}
\end{tsk}
\begin{tsk}
Та сама задача, але скільки часу займе її прогулянка?\\
Тобто, знайдемо $m_k$
\begin{eqnarray}
m_0 &=& 0\\
m_n &=& 0\\
m_k &=& 1 + \cfrac12 m_{k+1}  + \cfrac12 m_{k-1}
\end{eqnarray}
Розв’язавши це добро можна отримати:
\begin{equation}
m_k = k - k^2 + k \cb{n-1}
\end{equation}
\end{tsk}
\subsection{Домашнє завдання}
\begin{tsk}
Розв’язати обидві ці задачі, якщо ймовірність піти направо дорівнює $p$, а наліво $q$. Звісно, $p+q=1$.
\end{tsk}
\begin{tsk}
%переписати у Ганни
\end{tsk}

\section{Практика} \marginpar{\framebox{30.04.2014}}
\begin{tsk}
\begin{equation}
\xi = 2 \cdot w(1) - 3 w(2)+4w(3)+5
\end{equation}
\begin{equation}
f_\xi(x) = \cfrac{1}{\gamma \sqrt{2\pi}}e^{-\frac{\cb{x-a}^2}{2\sigma^2}}
\end{equation}
Знайдемо розподіл $\xi$:
\begin{equation}
\mEx = 2\mEt{w(1)} - 3\mEt{w(2)} +4\mEt{w(3)} + 5 = 5
\end{equation}
Знайдемо дисперсію:
\begin{multline}
\mDx = \mEt{\cb{\xi-\mEx}^2} = \mEt{\cb{2 \cdot w(1) - 3 w(2)+4w(3)}^2}  ={}\\{}= 4\mEt{w^2(1)}+9\mEt{w^2(2)} + 16\mEt{w^2(3)} -{}\\{}- 12\mEt{w(1)w(2)} - 16\mEt{w(1)w(3)} - 24\mEt{w(2)w(3)}  = 26
\end{multline}
Отримали:
\begin{equation}
f_\xi(x) = \cfrac{1}{\sqrt{52\pi}}e^{-\frac{\cb{x-5}^2}{52}}
\end{equation}
\end{tsk}
\begin{tsk}
%милий графік
\begin{equation}
\vec{w}(t) = \vect{w_1(t)\\w_2(t)}
\end{equation}
Знайдемо $\mEt{\nr{\vec w}(t)}$\\
\begin{equation}
\mEt{\nr{\vec w}(t)} = \mEt{\sqrt{w_1^2(t) + w_2^2(t)}}
\end{equation}
Розглянемо дещо окремо
\begin{equation}
f_{w_1(t)} (x) = f_{w_2(t)} (x) = \cfrac1{\sqrt{2t\pi}} e^{-\frac{x^2}{2t}}
\end{equation}
Отже, отримуємо інтеграл:
\begin{multline}
\mEt{\nr{\vec w}(t)} ={}\\{}= \int\intl_{\mR^2} \sqrt{x^2+y^2} \cfrac1{2\pi t} e^{-\frac{x^2+y^2}{2t}}\dx\dy = \intl_0^{2\pi} \dif\phi \intl_0^\infty \rho^2 \cfrac1{2\pi t} e^{-\frac{\rho^2}{2t}} \drh = {}\\{} = \intl_0^\infty e^{-\frac{\rho^2}{2t}} \cfrac{\rho^2}t \drh
\end{multline}
Розглянемо це окремо:
\begin{equation}
\intl_0^\infty e^{-\frac{\rho^2}{2t}} \cfrac{\rho^2}t \drh = \mdl{z = \cfrac{\rho^2}{2t}} = \intl_0^\infty e^{-z} 2z \cfrac{\sqrt t}{\sqrt{2z}} \dz = \sqrt{2t} G\cb{\cfrac32} = \cfrac{\pi t}{2}
\end{equation}
\end{tsk}
\begin{tsk}
$w(1),w(2),w(5)$ - спостерігається. Знайти оптимальну оцінку для $w(3)$.\\
Згадаємо формулу:
\begin{equation}
\heta = m_\eta + C_{\eta,\xi} \mD^{-1}_\xi \cb{\xi-m_\xi}
\end{equation}
\begin{eqnarray}
&\eta = w(3)\\
&\vxi = \vect{w(1)\\w(2)\\w(5)}
\end{eqnarray}
Відомо, що:
\begin{eqnarray}
&m_\eta = 0\\
&m_\xi = \vect{0\\0\\0}
\end{eqnarray}
\begin{multline}
\mDx ={}\\{}= \begin{pmatrix}
\mDt{w(1)} & \cdot & \cdot \\
\cdot & \mDt{w(2)} & \cdot\\
\cdot &\cdot & \mDt{w(5)} 
\end{pmatrix} = \begin{pmatrix}
\mEt{w^2(1)} & \mEt{w(1)w(2)} & \mEt{w(1)w(5)}\\
\mEt{w(2)w(1)} &\mEt{w^2(2)} & \mEt{w(2)w(5)}\\
\mEt{w(5)w(1)} & \mEt{w(5)w(2)} & \mEt{w^2(5)}
\end{pmatrix} ={}\\{} =
\begin{pmatrix}
1 & 1 & 1 \\
1 & 2 & 2\\
1 & 2 & 5
\end{pmatrix}
\end{multline}
\begin{multline}
C_{\eta,\xi} = \mEt{\cb{\eta-m_\eta}\mt{\cb{\xi-m_\xi}}} ={}\\{}= \mEt{\eta\cdot \mt{\xi}} = \mEt{w(3) \cdot \cb{w(1),w(2),w(5)}} = {}\\{} = \cb{1,2,5}
\end{multline}
Отже, в кінці отримали:
\begin{equation}
\heta = \cfrac23 w(2) + \cfrac13 w(5)
\end{equation}
\end{tsk} 
\begin{tsk}
\begin{equation}
\tilde{w}(t) = \system{t\cdot w\cb{\cfrac1t},t>0\\0,t=0}
\end{equation}
Довести, що $\tilde{w}(t)$ - вінеровський. Для цього мають виконуватися три умови:
\begin{enumerate}
\item $\tilde{w}(t)$ - гаусовський процес
\item $\mEt{\tilde{w}(t)} = 0$
\item $\mEt{\tilde{w}(t)\tilde{w}(s)} = \min\set{t,s}$
\end{enumerate}
Другий та третій пункти очевидні, просто перевіряються в лоб.
\begin{equation}
\vect{\tilde{w}(t_1)\\\vdots\\\tilde{w}(t_n)} = \vect{t_1w\cb{\cfrac1{t_1}}\\\vdots\\t_nw\cb{\cfrac1{t_n}}} = \begin{pmatrix}
t_1 & \ldots & 0 \\
\vdots&\vdots&\vdots\\
\ldots & \ldots & t_n
\end{pmatrix} \cdot \vect{w\cb{\cfrac1{t_1}}\\\vdots\\w\cb{\cfrac1{t_n}}}
\end{equation}
Отже, гаусовість зберігається.
\end{tsk}
\section{Практика}\marginpar{\framebox{14.05.2014}}
\begin{tsk}
	Знайти:
	\begin{equation}
		\mP\set{N(2)=3,N(3)=4,N(5)>6}
	\end{equation}
	\begin{multline}
		\mP\set{N(2)=3,N(3)=4,N(5)>6} ={}\\{}= \mP\set{N(2)=3,N(3)-N(2)=1,N(5)-N(3)>2} = {}\\ {} =\mP\set{N(2)=3}\cdot \mP\set{N(3)-N(2)=1} \cdot \mP\set{N(5)-N(3)>2} ={}\\{}= \cfrac{e^{-2\la} \cb{2\la}^3}{3!} \cdot \cfrac{e^{-\la} \la }{1!} \cb{1 - \cfrac{e^{-2\la}\cb{2\la}^0}{1} - \cfrac{e^{-2\la}\cb{2\la}^1}{1} - \cfrac{e^{-2\la}\cb{2\la}^2}{2}} ={}\\{}= \cfrac{e^{-3\la} 4 \la^4}{3} \cb{1 - e^{-2\la} - 2\la e^{-2\la} - 2\la^2 \cb{-2\la}}
	\end{multline}
\end{tsk}
\begin{tsk}
	Знайти
	\begin{equation}
		\mP\set{N(t)\div2}
	\end{equation}
	\begin{multline}
		\mP\set{N(t)\div2} = {}\\{} = \mP\set{\exists k\in N: N(5) = 2k} = \suml_{k=0}^\infty \mP\set{N(5)= 2k}  = \suml_{k=0}^\infty e^{-5\la} \cfrac{\cb{5\la}^{2k}}{2k!} = {} \\ {} = e^{-5\la} \suml_{k=0}^\infty  \cfrac{\cb{5\la}^{2k}}{2k!} = e^{-5\la} + \ch 5\la ={}\\{} = \cfrac12 + \cfrac12 e^{-10\la}
	\end{multline}
\end{tsk}
\begin{tsk}
	Знайти
	\begin{equation}
		\mP\set{N(1)+N(3)+N(5) = N(2)+N(4)+N(6)}
	\end{equation}
	\begin{multline}
		\mP\set{N(1)+N(3)+N(5) = N(2)+N(4)+N(6)} ={}\\{}= \mP\set{N(1)=N(2),N(3)=N(4),N(5)=N(6)} ={}\\{}= \mP\set{N(1)-N(2)=0}\cdot \mP\set{N(3)-N(4)=0} \cdot \mP\set{N(5)-N(6)=0}  ={}\\{} = e^{-3\la}
	\end{multline}
\end{tsk}
\begin{tsk}
	Знайти
	\begin{equation}
		\mP\set{N(1)+N(3)+N(5) +10 = N(2)+N(4)+N(6)}
	\end{equation}
	\begin{multline}
		\mP\set{N(1)+N(3)+N(5) +10 = N(2)+N(4)+N(6)} = {} \\ {} = \mP\set{\cb{N(2)-N(1)} + \cb{N(4) - N(3)} + \cb{N(5)-N(6)} = 10} ={}\\{} = \mP\set{\xi_1+\xi_2+\xi_3 = 10},\xi_1,\xi_2,\xi_3\sim Pois(\la)
	\end{multline}
	\begin{equation}
		\mP\set{\xi_1+\xi_2+\xi_3 = 10} = \mP\set{\eta = 10},\eta\sim Pois(3\la)
	\end{equation}
	\begin{equation}
		\mP\set{\eta=10} = e^{-3\la} \cfrac{\cb{3\la}^10}{10!}
	\end{equation}
\end{tsk}
\subsection{Домашнє завдання}
\begin{tsk}
	Знайти коваріаціну функцію процесу $X(t)$, де
	\begin{equation}
		X(t) = N(t+1)-N(t)
	\end{equation}
\end{tsk}
\begin{tsk}
	Знайти
	\begin{equation}
		f_{\tau\diagup_{N(1)=1}} (x)
	\end{equation}
	Де $\tau$ - момент появи першої події у потоку Пуассона
\end{tsk}


\section{Література}
\end{document} 
