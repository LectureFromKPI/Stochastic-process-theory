\section{Практика}\marginpar{\framebox{19.03.2014}}
\begin{tsk}
\begin{tikzpicture}
\node[] (c1) {Саша};
\node[right=of c1] (c2) {Паша};
\node[right=of c2] (c3) {Даша};
\path[->]
(c1) edge node[above]{1} (c2)
(c2) edge node[above]{1}(c3)
(c3) edge[loop right] node[right]{1/2}(c3)
(c3) edge[bend left] node[below]{1/2} (c1)
;
\end{tikzpicture}
\begin{equation}
\kvp^{(0)} = \cb{1,0,0}
\end{equation}
Мінімальне $k=4$.\\
Запишемо систему:
\begin{equation}
\system{\kv{\Pi}\mP = \kv\Pi\\\Pi_1+\Pi_2+\Pi_3=1}
\end{equation}
Розпишемо цю систему:
\begin{equation}
\system{
\cfrac12\Pi_3 = \Pi_1\\ 
\Pi_1=\Pi_2\\
\Pi_2+\cfrac12\Pi_3 = \Pi_3\\
\Pi_1+\Pi_2+\Pi_3=1
}
\end{equation}
Отримали розв’язок
\begin{equation}
\kv\Pi = \cb{\cfrac14,\cfrac14,\cfrac12}
\end{equation}
Знайдемо власні числа нашої матриці $\mP$.\\
\begin{equation}
\det\cb{\mP-\la I} = \begin{vmatrix}
-\la & 1 & 0\\
0 & -\la & 1 \\
\cfrac12 & 0 & \cfrac12-\la
\end{vmatrix} = -\la^3+\cfrac{\la^2}2+ \cfrac12
\end{equation}
Розв’язавши це рівняння отримуємо:
\begin{equation}
\la = \cb{1,\cfrac14\cb{-1+i\sqrt7},\cfrac14\cb{-1-i\sqrt7}}
\end{equation}
Загалом, все сумно. Тоді спробуємо щось зробити більш адекватне. Але ні, не спробували.
\end{tsk}
\begin{tsk}
Знайте математичне сподівання момента першого повернення до Саші.\\
$\tau$ - момент першого повернення до Саші.\\
\begin{tabular}{|c|c|c|c|c|c|c|}
\hline
$\tau$ & 1 & 2 & 3  & 4 & 5 & $\ldots$\\
\hline
$\mP$ & 0 & 0 & 1/2 & 1/4 & 1/16 & $\ldots$ \\
\hline
\end{tabular}\\
\begin{equation}
\mEt\tau = \suml_{i=3}^\infty p_i = \suml_{i=3}^\infty 2^{-(k-2)} i = \suml_{l=0}^\infty 2^{-(l+1)} (l+3) = 1+3 = 4
\end{equation}
\end{tsk}
\subsection{Домашнє завдання}
\begin{tsk}
Маша підкидає монетку нескінченну кількість разів. Яка ймовірність того, що три решки підряд з’являться раніше, ніж два герби підряд.\\
Розглянемо таку схему:\\
\begin{tikzpicture}
\node[] (s1) {1.РГ};
\node[below right=of s1] (s11) {5.РГГ};
\node[below left=of s1] (s12) {2.ГР};
\node[below=of s12] (s122){3.ГРР};
\node[below=of s122] (s1222) {4.РРР};
\node[above left=of s1] (s) {Старт};
\path[->]
(s1) edge node[above]{1/2} (s11)
(s12) edge[bend left] node[above]{1/2} (s1)
(s1) edge node[below]{1/2} (s12)
(s12) edge node[left]{1/2} (s122)
(s122) edge[bend right]node[left]{1/2}(s1)
(s122) edge node[left]{1/2} (s1222)
(s) edge node[right]{1/2} (s1)
(s) edge node[left]{1/2} (s12)
;
\end{tikzpicture}\\
Отримали ймовірності: $h_1 = \cfrac15;h_2 = \cfrac25;h_3 = \cfrac35;h_4=1$\\
Тоді, за формулою повної ймовірності:
\begin{equation}
\cfrac12 \cdot h_1 + \cfrac12 \cdot h_2 = \cfrac3{10}
\end{equation}
\end{tsk}
\begin{tsk}
\begin{center}
\begin{tikzpicture}[node distance=2cm]
\node[] (c1) {2.Дом};
\node[below right=of c1] (c2) {1.Поле};
\node[above right=of c2] (c3) {3.Ручей};
\node[below=of c2] (c4) {5.Обрив};
\node[right=of c3] (c5) {4.Крокодил};
\node[above=of c1] (c6) {6.М'ясокобінат};
\path[->]
(c1) edge[bend right] node[below left]{3/5} (c2)
(c1) edge[bend left] node[above]{1/5} (c3)
(c1) edge node[left]{1/5} (c6) 
(c2) edge[bend right] node[below right]{3/5} (c3)
(c2) edge node[above right]{1/5} (c1)
(c2) edge node[left]{1/5} (c4)
(c3) edge node[above]{2/5} (c1) 
(c3) edge node[above left]{2/5} (c2)
(c3) edge node[above]{1/5} (c5)
;
\end{tikzpicture}
\end{center}
\begin{itemize}
\item Спочатку корова у домі, обчислити ймовірність кожної смерті на нескінченності.
\item Знайти математичне сподівання смерті.
\end{itemize}
\begin{equation}
h_i = \mP\set{\xi_\infty=4\setminus\xi_0=i}
\end{equation}
\begin{eqnarray}
h_4 &=& 1\\
h_5 &=& 0\\
h_6 &=& 0\\
h_1 &=& \cfrac15 h_2 + \cfrac15 \cdot 0 + \cfrac35 h_3\\
h_2 &=& \cfrac35 h_1 + \cfrac15 \cdot 0 + \cfrac15 h_3 \\
h_3 &=& \cfrac15 \cdot 1 + \cfrac25 h_1 + \cfrac25 h_2
\end{eqnarray}
Розв’яжемо цю милу систему рівнянь.\\
Отримали: $h_1 = 0.32;h_2 = 0.28;h_3 = 0.44$\\
Тепер знайдемо математичні сподівання:
\begin{eqnarray}
m_4 &=& 0\\
m_5 &=& 0\\
m_6 &=& 0\\
m_2 &=& 1 + \cfrac15\cdot 0 +\cfrac15 \cdot m_3 + \cfrac35 \cdot m_1\\
m_1 &=& 1+ \cfrac15 \cdot 0 + \cfrac35 \cdot m_3 +  \cfrac15 \cdot m_2\\
m_3 &=& 1 + \cfrac15 \cdot 0 + \cfrac25 \cdot m_2 + \cfrac25 \cdot m_1
\end{eqnarray}
Розв’яжемо це рівняння:
\begin{equation}
m_1 = m_2 = m_3 = 5
\end{equation}
\end{tsk}