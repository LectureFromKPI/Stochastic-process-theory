\section{Практика}\marginpar{\framebox{05.03.2014}}
\begin{tsk}
$\xi\sim Pois\cb{\la}$ - кількість студентів. \\
Студент здає іспит з ймовірністю $p$ і не здає з ймовірністю $1-p$.\\
Потрібно довести, що кількість тих, хто склад іспит також розподілена $\sim Pois\cb{\la p}$.
\begin{multline}
\mP\set{\eta=k} = \suml_{i=k}^\infty \mP\set{\xi=i}\cdot \mP\set{\eta=k\setminus\xi=i} = \suml_{i=k}^\infty \cfrac{e^{-\la}\la^i}{i!} C^k_i p^k \cb{1-p}^{i-k} ={}\\{}= e^{-\la} p^k \suml_{i=k}^\infty \cfrac{\la^i}{i!} C^k_i \cb{1-p}^{i-k} = \begin{vmatrix}
j=i-k
\end{vmatrix} = e^{-\la} p^k \suml_{j=0}^\infty \cfrac{\la^{j+k}}{(j+k)!} C^k_{j+k} \cb{1-p}^{j} = {} \\ {}= \cfrac{e^{-\la}\la^k}{k!} p^k \suml_{j=0}^\infty \cfrac{\la^j}{j!} \cb{1-p}^j =  \cfrac{e^{-\la}\la^k}{k!} p^k \cdot e^{\la(1-p)} = \cfrac{e^{-p\la} \cb{\la p}^k}{k!}
\end{multline}
\end{tsk} 
\begin{tsk}
$\veps=\vect{\eps_1\\\eps_2\\\eps_3\\\eps_4\\\eps_5}$\\
Це \textbf{вектор білого шуму}, тобто:
\begin{itemize}
\item $\mEt{\eps_i}=0,\forall i$;
\item $\mDt{\eps_i}=1,\forall i$;
\item $\mEt{\eps_i\eps_j}=0,\forall i,j:i\neq j$.
\end{itemize}
Будуємо з нього інший вектор:
\begin{equation}
\vu = \vect{\eps_1+\eps_2+\eps_3+\eps_4+\eps_5\\2\eps_1-\eps_2-\eps+4+2\eps_5\\\eps_1+3\eps_3+\eps_5\\ \eps_2-\eps_1\\ \eps_5-\eps_4}
\end{equation}
Спостерігаються координати з непарними номерами, а знайти оцінку координат з парними номерами.
\begin{eqnarray}
\vxi=\vect{\eps_1+\eps_2+\eps_3+\eps_4+\eps_5\\ \eps_1+3\eps_3+\eps_5 \\ \eps_5-\eps_4}\\
A_{\vxi} = \begin{pmatrix}
1 & 1 & 1 & 1 & 1 \\
1 & 0 & 3 & 0 & 1\\
0 & 0 & 0 & -1 & 1
\end{pmatrix}\\
\veta = \vect{2\eps_1-\eps_2-\eps+4+2\eps_5\\ \eps_2-\eps_1}\\
B_{\vxi} = \begin{pmatrix}
2 & -1 & 0 & -1 & 2 \\
-1 & 1 & 0 & 0 & 0
\end{pmatrix}
\end{eqnarray}
Запишемо формулу оцінки:
\begin{equation}
\hveta = \vm_{\veta} + C_{\veta,\vxi} \iD_{\vxi}^{-1} \cb{\vxi-\vm_{\vxi}}
\end{equation}
З умови відомо, що:
\begin{equation}
\vm_{\veta}=\vec 0;\vm_{\vxi} = \vec 0
\end{equation}
Знайдемо дисперсійну матрицю:
\begin{multline}
\iD_{\vxi} = \mEt{\cb{\vxi-\mEvx}\mt{\cb{\vxi-\mEvx}}} = \mEt{\cb{A\veps-\mEt{A\veps}}\mt{\cb{A\veps-\mEt{A\veps}}}} = {} \\ {} =  \mEt{A\cb{\veps-\mEt{\veps}}\mt{A\cb{\veps-\mEt{\veps}}}} = \mEt{A\cb{\veps-\mEt{\veps}}\mt{\cb{\veps-\mEt{\veps}}}\mt{A}} = {} \\ {} = A\underbrace{\mEt{\cb{\veps-\mEt{\veps}} \mt{\cb{\veps-\mEt{\veps}}}}}_{\iD_{\veps}}\mt{A} = A\mt A
\end{multline}
Здогадаємося по аналогії до формули:
\begin{equation}
C_{\veta,\vxi} = B \mt A
\end{equation}
Отже, отримали:
\begin{equation}
\hveta = B\mt A \cb{A \mt A}^{-1}\vxi = \begin{pmatrix}
\frac{17}{55} & \frac1{11} & \frac{16}{11}\\
\frac2{11}& -\frac{2}{11} & \frac1{11}
\end{pmatrix}\cdot\vect{\xi_1\\ \xi_2\\ \xi_3}
\end{equation}
\end{tsk}
\begin{tsk}
Саша та Маша стріляють в лося. Кожен робить по 10 пострілів, при чому навіть у мертве тіло. Ймовірність потрапляння Саші $0.9$, а Маши $0.5$. Спостерігається загальна кількість влучень, побудувати лінійну оцінку для кількості потраплянь Саші.\\
%Хтось запишіть сюди задачу
\end{tsk}
\begin{tsk}
Випадкова величина $X \sim U(\bb{0,\cfrac\pi2})$. Спостерігається $\sin$, оцінити $\cos$.
\begin{eqnarray}
&\xi = \sin X;\eta = \cos X\\
&\mEe = \cfrac2\pi \\
&\mEx = \cfrac2\pi\\
&\mEt{\xi^2} = \cfrac12\\
&\mDx=\cfrac12 - \cfrac4{\pi^2}\\
&K_{\xi,\eta} = \cfrac1\pi - \cfrac4{\pi^2}\\
&\heta = \cfrac2\pi +\cb{\cfrac1\pi -\cfrac4{\pi^2}}
\end{eqnarray}
%Допишіть хтось
\end{tsk}
\begin{tsk}
Розподіл числа потомків є геометричним з параметром $p$.\\
Знайти ймовірність виродження гіллястого процесу.
%Перепишіть хтось
\end{tsk}
\subsection{Домашнє завдання}
\begin{tsk}\label{pr:2:1}
Маша стріляє в круглу ціль радіусу 1 метр. Точка потрапляння рівномірно розподілена в цьому колі. Спостерігається відхилення від центра по горизонталі. Знайти оптимальну лінійну оцінку відхилення по вертикалі. Відхилення обчислюється через модулі.\\
$\mdl{\xi}$ - відстань по горизонталі.\\
$\mdl{\eta}$ - відстань по вертикалі.\\
\begin{equation}
\mdl{\heta} = m_{\mdl{\eta}} + C_{\mdl{\eta},\mdl{\xi}} D^{-1}\cb{\mdl{\xi}-m_{\mdl\xi}}
\end{equation}
%очевидно, зробити самому потім
\end{tsk}
\begin{tsk}
Спостерігається обидва відхилення в задачі \eqref{pr:2:1}. Оцінити відстань від точки потрапляння до центру мішені. 
\end{tsk}
\begin{tsk}
Теж сама задача, що й \eqref{pr:2:1}, але в нас тепер немає мішені. Координати точки потрапляння це незалежні гаусовські величини. Спостерігаються відхилення від вісей, оцінити відстань від точки потрапляння до початку координат.
\end{tsk}
\begin{tsk}
Число потомків має пуасонівський розподіл. Яким має бути параметр цього розподіл, щоб ймовірність виродження дорівнювала $\cfrac12$.
\end{tsk}