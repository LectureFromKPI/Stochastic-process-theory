\chapter{Випадкові процеси} 
\section{Основні поняття}\marginpar{\framebox{09.04.2014}}
\textbf{Випадковий процес} $\cb{X(t),t\in T}$ - це сукупність випадкових величин $X(t)$, індексованих параметром $t\in T$, які задані на спільному ймовірнісному просторі $\cb{\Omg,\iF,\mP}$\\
Розглянемо деякі випадки:
\begin{itemize}
\item $T={1}$ - випадкова величина;
\item $T={1,2,\ldots,n}$ - випадковий вектор;
\item $T=\mN$ - випадкова послідовність. Також, цей випадок допускає будь-яку злічену множину;
\item $T=\bb{a,b},\cb{a,b},\ldots$ - тобто, деякий інтервал. І це буде називатися просто випадковий процес.
\item $T=\mR^d$ - випадкове поле.
\end{itemize}
Фактично, параметр $t$ можна ототожнювати з часом.\\
Отже, випадковий процес $X$ - це функція від двох змінних $X\cb{t,\omg}$. Тоді, з цієї точки зору, можна сказати, що $X:T\times\Omg\to \mR$. Оскільки нам необхідна випадковість, то необхідно вимагати, щоб:
\begin{equation}
\forall t\in T:X(t,\cdot) - \text{вимірна, відсно }\iF\setminus \iB\cb{\mR}
\end{equation}
$X(t,\cdot)$ - випадкове значення нашого процесу $X$ в момент $t$.\\
$X(\cdot,\omg)$ - \textbf{траєкторія} або \textbf{реалізація} випадкового процесу $X$.\\
\begin{exs}
В нас є параметрична множина $T=\cb{0,90}$.\\
В момент $\tau$ на пару приходить Павло. $\tau\sim U\cb{0,90}$.\\
Тепер введемо деякий випадковий процес: $X(t)$ - кількість Павлів на парі у момент $t$.\\
\begin{equation*}
X(t)=\system{
0, t<\tau\\
1, t\geq \tau
}
\end{equation*}
%Намалюємо траєкторію:\\
%Намалювати графік цієї штуки
\end{exs}
\subsection{Основна характеристика випадкового процесу}
Основна характеристика випадкового процесу це \textbf{система скінченновимірних розподілів}.
\begin{itemize}
\item Одновимірні функції розподілу:\\
\begin{equation}
F_X(t,x) = \mP\set{X(t)<x}
\end{equation}
Одновимірний розподіл:
\begin{equation}
P_t(B) = \mP\set{X(t)\in B};B\in\iB\cb{\mR^1}
\end{equation}
\item Двовимірні функції розподілу:\\
\begin{equation}
F_X(t_1,t_2,x_1,x_2) = \mP\set{X(t_1)< x_1 \cap X(t_2) < x_2}
\end{equation}
Двовимірний розподіл:
\begin{equation}
P_{t_1,t_2} (B) = \mP\set{(X(t_1),X(t_2)}\in B;B\in\iB\cb{\mR^n}
\end{equation}
І так далі можна ввести $n$-вимірні функції розподілу і $n$-вимірний розподіл.
\end{itemize}
\subsubsection{Система випадкових розподілів}
Потрібно задати усю систему - $\forall n\geq 1,\forall t_1,\ldots,t_n \in T$.
\begin{exs}
\begin{equation*}
X(t) = \system{0,t<\tau\\1,t\geq \tau}
\end{equation*}
$t$ - фіксуємо.\\
\begin{center}
\begin{tabular}{|c|c|c|}
$X(t)$ & 0 & 1 \\
\hline
$\mP$ & $1-\cfrac t{90}$ & $\cfrac t{90}$
\end{tabular}
\end{center}
Двовимірний розподіл: фіксуємо $t_1,t_2(t1\leq t_2)$.\\
\begin{center}
\begin{tabular}{|c|c|c|}
$X(t_2)\setminus X(t_1)$ & 0 & 1 \\
\hline
0 & $1 - \cfrac t{90}$ & 0 \\
\hline
1 & $\cfrac{t_2-t_1}{90}$ & $\cfrac{t_1}{90}$
\end{tabular}
\end{center}
$X(t_1) = X(t_2) =0$\\
\end{exs}
\subsection{Властивості скінченновимірного розподілу}
Розглянемо $P_{t_1,\ldots,t_n} (B),B\in\iB\cb{\mR^n}$.\\
%Знайти у когось тут властивосты
\begin{teor}[Теорема Колмагорова]
Нехай задана система скінченновимірних розподілів: $P_{t_1,\ldots,t_n}(B_1\times \ldots\times B_n),\forall n\in\mN$:\\
яка задовольняє попереднім умовам %тим, які потрібно у когось знайти
Тоді, на деякому ймовірнісному просторі $\cb{\Omg,\iF,\mP}$ існує випадковий процес $X(t,\omg)$, який має саме цю систему скінченних розподілів.
\end{teor}
\section{$\mL_2$-процеси}
$X$ - це \textbf{$\mL_2$-процес}, якщо:
\begin{equation}
\forall t:\mEt{\mdl{X(t)}}<\infty \vee \mEt{\mdl{X(t)}^2}<\infty
\end{equation}
Тоді можемо ввести \textbf{математичне сподівання}:
\begin{equation}
m(t) = \mEt{X(t)},t\in T
\end{equation}
Це може бути будь-яка функція, без будь-яких обмежень.\\
\textbf{Кореляційна функція}:
\begin{equation}
C_x(s,t) = \mEt{\cb{X(s)-m(s)}\overline{\cb{X(t)-m(t)}}} = \mEt{X(s)\overline{X(t)}} - \cb{\mEt{X(s)}}\overline{\cb{\mEt{X(t)}}}
\end{equation}
Чому вона існує:
\begin{equation}
\mdl{\mEt{\cb{X(s)-m(s)}\overline{\cb{X(t)-m(t)}}}} \leq \sqrt{\mEt{\mdl{X(s)-m(s)}^2} + \mEt{\mdl{X(t)-m(t)}}^2} < \infty
\end{equation}
\subsection{Властивості кореляційної функції}
\begin{teor}[Ермітовість]
$C_x(s,t) = \overline{C_x(t,s)}$ 
\end{teor}
\begin{proof}
Позначимо: $X_0(t) = X(t) - m(t)$\\
\begin{equation*}
C_x(t,s) = \mEt{X_0(t)\overline{X_0(s)}} = \overline{\mEt{X_0(s)\overline{X_0(t)}}} = \overline{C_x(s,t)}
\end{equation*}
\end{proof}
\begin{teor}
Розглянемо $\vec X = \vect{X(t_1)\\ \vdots \\ X(t_n)}$\\
$C_{\vec X} = \mdl{\mdl{C_x(t_i,t_j)}}_{i,j=1,\ldots,n}$ \\
Оскільки це дуже схоже на кореляційну матрицю, вона отримує деяку її властивість:
$\forall t_1,\ldots,t_n:\mdl{\mdl{C_x(t_i,t_j)}}_{i,j=1,\ldots,n}$ - невід’ємно визначена.
\end{teor}
А це означає, що така квадратична форма:
\begin{equation}
\cb{\mdl{\mdl{C_x(t_i,t_j)}}_{i,j=1,\ldots,n} \vec b,\vec b} \geq 0,\forall \vec b \in \mC^n
\end{equation}
Або, якщо переписати у вигляді суми:
\begin{equation} \label{tr:8:1}
\suml_{i=1}^n \suml_{j=1}^n C_x\cb{t_i,t_j} b_i\overline{b_j} \geq 0,\forall b_1,\ldots,b_n\in\mC
\end{equation}
Отже, кореляційна функція завжди ермітово симетрична і задовольняє властивість \eqref{tr:8:1}. Функції, що задовольняють умові \eqref{tr:8:1}, то вони називаються \textbf{невід’ємно визначеними}.
\begin{teor}
Якщо функція є ермітово симетричною та невідємно визначеною, то вона є кореляційною функцією.
\end{teor}
\begin{proof}
Фіксуємо $n\geq 1$.\\
Фіксуємо $t_1,\ldots,t_n\in T$.\\
Тоді матриця $С = \mdl{\mdl{C_x(t_i,t_j)}}_{i,j=1,\ldots,n} $ - є невід’ємно визначеною та ермітовою.\\
Тоді, $P_{t_1,\ldots,t_n} (B)=$ розподіл Гаусовського вектора з кореляційною матрицею $\mdl{\mdl{C_x(t_i,t_j)}}_{i,j=1,\ldots,n}$\\
Тоді, за теоремою Колмагорова, можна побудувати процес.
\end{proof}
\begin{exs}
$T = \bb{0,+\infty}$\\
$C(s,t) = \min\set{s,t}$\\
\begin{itemize}
\item $\min{s,t} = \min\set{t,s}$ - симетричність;
\item $\mdl{\mdl{\min\set{s,t}}}_{i,j=1,\ldots,n}$ - невід’ємна визначна (раніше було).
\end{itemize}
\end{exs}
\begin{exs}
\begin{equation}
X(t) = \system{0, t<\tau \\ 1, t>\tau}
\end{equation}
\begin{equation}
m(t) = \mEx{X(t)} = 0 \mP\set{X(t)=0} + 1\cdot \mP\set{x(t) = 1} = \cfrac t{90}
\end{equation}
Тепер шукаємо кореляційну функцію:
\begin{equation}
C(s,t) = \mEt{X(t)\cdot X(s)} - \mEt{X(t)} \mEt{X(s)} = \cfrac{\min\set{s,t}}{90} - \cfrac{st}{8100}
\end{equation}
\end{exs}